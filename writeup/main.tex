\documentclass{article}
\usepackage[utf8]{inputenc}
\usepackage{amsmath, amssymb}
\usepackage{hyperref, float}
\usepackage{booktabs}
\usepackage{pdflscape}
\usepackage{caption, subcaption}
\usepackage{graphicx}
        

\title{San Diego Calibration Results}
\author{Sharad Vikram}
\date{\today}

\begin{document}

\maketitle

\section{Data}

We have been collecting data from nine boards
from three sites in southern California.
\begin{enumerate}
    \item El Cajon
    \item Donovan
    \item Shafter
\end{enumerate}
We have split up the boards and rotated the boards
between locations every two weeks (see \autoref{tab:board-rotations}).

\begin{table}[H]
\centering
\begin{tabular}{l|lll}
                  & \textbf{Round 1} & \textbf{Round 2} & \textbf{Round 3} \\ \hline
\textbf{Board 17} & N/A & El Cajon         & Shafter          \\
\textbf{Board 19} & Donovan          & El Cajon         & Shafter          \\
\textbf{Board 21} & Donovan          & El Cajon         & Shafter          \\ \hline
\textbf{Board 11} & El Cajon         & Shafter          & Donovan          \\
\textbf{Board 12} & El Cajon         & Shafter          & Donovan          \\
\textbf{Board 13} & El Cajon         & Shafter          & Donovan          \\ \hline
\textbf{Board 15} & Shafter          & Donovan          & El Cajon         \\
\textbf{Board 18} & Shafter          & Donovan          & El Cajon         \\
\textbf{Board 20} & N/A & Donovan          & El Cajon        
\end{tabular}
\caption{Board locations for each round}
\label{tab:board-rotations}
\end{table}

We do not have CO data for Shafter and Donovan, so we will focus only on
O3 and NO2.

\section{Distributions}
In this section, we describe
and visualize the distributions
of various values in the data.

\subsection{Environment}

\begin{figure}
\centering
\includegraphics[width=0.5\textwidth]{results/distributions/temperature.png}
\caption{Temperature distribution based on
location}
\label{fig:temperature}
\end{figure}

\begin{figure}
\centering
\includegraphics[width=0.5\textwidth]{results/distributions/humidity.png}
\caption{Absolute humidity distribution based on
location}
\label{fig:humidity}
\end{figure}

\begin{figure}
\centering
\begin{subfigure}{0.32\textwidth}
\includegraphics[width=\textwidth]{results/distributions/location_donovan_temperature.png}
\caption{Donovan}
\end{subfigure}
\begin{subfigure}{0.32\textwidth}
\includegraphics[width=\textwidth]{results/distributions/location_elcajon_temperature.png}
\caption{El Cajon}
\end{subfigure}
\begin{subfigure}{0.32\textwidth}
\includegraphics[width=\textwidth]{results/distributions/location_shafter_temperature.png}
\caption{Shafter}
\end{subfigure}
\caption{Temperature at locations}
\label{fig:temperature-locations}
\end{figure}

\begin{figure}
\centering
\begin{subfigure}{0.32\textwidth}
\includegraphics[width=\textwidth]{results/distributions/location_donovan_humidity.png}
\caption{Donovan}
\end{subfigure}
\begin{subfigure}{0.32\textwidth}
\includegraphics[width=\textwidth]{results/distributions/location_elcajon_humidity.png}
\caption{El Cajon}
\end{subfigure}
\begin{subfigure}{0.32\textwidth}
\includegraphics[width=\textwidth]{results/distributions/location_shafter_humidity.png}
\caption{Shafter}
\end{subfigure}
\caption{Humidity at locations}
\label{fig:humidity-locations}
\end{figure}

\begin{figure}
\centering
\begin{subfigure}{0.32\textwidth}
\includegraphics[width=\textwidth]{results/distributions/round1_temperature.png}
\caption{Round 1}
\end{subfigure}
\begin{subfigure}{0.32\textwidth}
\includegraphics[width=\textwidth]{results/distributions/round2_temperature.png}
\caption{Round 2}
\end{subfigure}
\begin{subfigure}{0.32\textwidth}
\includegraphics[width=\textwidth]{results/distributions/round3_temperature.png}
\caption{Round 3}
\end{subfigure}
\caption{Temperature over rounds}
\label{fig:temperature-rounds}
\end{figure}

\subsection{Pollutant values}

\section{Basic calibration results}

A calibration model takes in sensor readings and environment
variables and outputs pollutant levels. In this basic setup,
we train a model for each board.
We aim to train models that are robust after moving location.

\begin{table}[H]
\begin{tabular}{|l|l|}
\hline
\textbf{Level 0} & Train on location A and test on location A \\ \hline
\textbf{Level 1} & Train on location A and test on location B \\ \hline
\textbf{Level 2} & Train on location A and B and test on location C \\ \hline
\textbf{Level 3} & Train on location A, B, and C and test on location A \\ \hline
\end{tabular}
\caption{Description of different types of benchmarks.}
\label{tab:levels}
\end{table}

We benchmark four different models: linear regression (linear), random forest regressors based on \cite{subu}(Subu),
a 2-layer neural network (NN[2]), and a 4-layer neural network (NN[4]). The ideal model will
both predict pollutant levels accurately and
generalize across locations.

To benchmark, we first take our datasets (25 total, see \autoref{tab:board-rotations}), and partition each into training and test sets (20\% reserved for testing).
We perform several types of benchmarks,
each to learn about the transferrability of each model (see \autoref{tab:levels}).
In general, we expect Level 0 and Level 3 performance to be the best, as they involve training and testing on data from the same distribution. Furthermore, we expected Level 2 to have lower error than Level 1, because Level 2 is trained on more data and a wider distribution of data (two locations vs one location).
If a model's Level 1 and Level 2 error are close to Level 0 and Level 3, then the model transfers well. Otherwise, the model overfits to its location.


These raw results are in \autoref{sec:simpleresults}. 
We split results into train vs. test
results, where we expect train performance
to be better than test.
Overall, we see that random forests have the lowest Level 0 and Level 3 error. This is consistent with results we see in \cite{subu}. 

\begin{figure}[H]
\centering
\begin{subfigure}{0.45\textwidth}
\includegraphics[width=\textwidth]{results/linear/no2.png}
\caption{NO2}
\end{subfigure}
\begin{subfigure}{0.45\textwidth}
\includegraphics[width=\textwidth]{results/linear/o3.png}
\caption{O3}
\end{subfigure}
\caption{Results for linear regression. Error is in parts per billion}
\label{fig:results-linear}
\end{figure}

\begin{figure}[H]
\centering
\begin{subfigure}{0.45\textwidth}
\includegraphics[width=\textwidth]{results/nn-2/no2.png}
\caption{NO2}
\end{subfigure}
\begin{subfigure}{0.45\textwidth}
\includegraphics[width=\textwidth]{results/nn-2/o3.png}
\caption{O3}
\end{subfigure}
\caption{Results for NN[2]. Error is in parts per billion}
\label{fig:results-nn2}
\end{figure}

\begin{figure}[H]
\centering
\begin{subfigure}{0.45\textwidth}
\includegraphics[width=\textwidth]{results/nn-4/no2.png}
\caption{NO2}
\end{subfigure}
\begin{subfigure}{0.45\textwidth}
\includegraphics[width=\textwidth]{results/nn-4/o3.png}
\caption{O3}
\end{subfigure}
\caption{Results for NN[4]. Error is in parts per billion}
\label{fig:results-nn4}
\end{figure}

\begin{figure}[H]
\centering
\begin{subfigure}{0.45\textwidth}
\includegraphics[width=\textwidth]{results/subu/no2.png}
\caption{NO2}
\end{subfigure}
\begin{subfigure}{0.45\textwidth}
\includegraphics[width=\textwidth]{results/subu/o3.png}
\caption{O3}
\end{subfigure}
\caption{Results for Subu. Error is in parts per billion}
\label{fig:results-subu}
\end{figure}


We observe when comparing Level 1 error difference (Level 1 train minus Level 1 test), random forests suffer great
drops in performance.
This hints that RFs are overfitting to the training data, even if they
report the lowest test error for Level 0 and Level 3.  See \autoref{fig:generalization} for 
details.

\begin{figure}[H]
\centering
\begin{subfigure}{0.45\textwidth}
\includegraphics[width=\textwidth]{results/no2mae_diff.png}
\caption{NO2}
\end{subfigure}
\begin{subfigure}{0.45\textwidth}
\includegraphics[width=\textwidth]{results/o3mae_diff.png}
\caption{O3}
\end{subfigure}
\caption{Level 1 difference plots. Train minus test errors for various models. A smaller value means that the models transfer better.}
\label{fig:generalization}
\end{figure}

\section{Neural representation learning}

We now present split-neural network results: we split
up calibration into two stages, a sensor model
and a pollutant model, which we will call $s_i$ and $c$
respectively.
Given a sensor readings $x$ from board $i$,
and environment readings $e$
we obtain a calibrated reading $y$ by simply passing it through
the sensor model, then the pollutant model, i.e.
\begin{align*}
    y = c(s_i(x), e)
\end{align*}
We can learn individual sensor models for each board,
but the pollutant model is shared across boards. This allows
us to pool data across boards to learn the pollutant model.
Furthermore, environment variables are only 
included in the pollutant model, which hopefully enables
a stronger fit with a very complex pollutant model.

Each $s_i(x)$ outputs a ``sensor representation'', which is chosen
to be some fixed dimension $d$. We hope that the sensor representation
contains the minimal information to produce calibrated readings.

We experiment with each $s_i$ being a linear regression model,
and $c$ being a deep neural network (two layers, 100 width ReLU). 

\appendix
\setcounter{table}{0}
\renewcommand{\thetable}{\Alph{section}.\arabic{table}}

\section{Raw results for simple calibration models}
\label{sec:simpleresults}

\subsection{Benchmarks for linear regression}
\label{sec:results-lr}

\begin{table}[H]
\centering
\scriptsize
\begin{tabular}{lllrrrr}
\toprule
{} &             Model &  Testing Location &   NO2 MAE &    O3 MAE &  NO2 CvMAE &  O3 CvMAE \\
\midrule
11 &  (1, donovan, 19) &  (1, donovan, 19) &  2.817624 &  3.748897 &   0.669465 &  0.098983 \\
13 &  (1, donovan, 21) &  (1, donovan, 21) &  2.532854 &  4.071486 &   0.601804 &  0.107501 \\
10 &  (1, elcajon, 11) &  (1, elcajon, 11) &  2.042837 &  5.212672 &   0.380986 &  0.137718 \\
18 &  (1, elcajon, 12) &  (1, elcajon, 12) &  2.196888 &  5.192453 &   0.409716 &  0.137184 \\
5  &  (1, elcajon, 13) &  (1, elcajon, 13) &  2.087088 &  4.383210 &   0.389238 &  0.115804 \\
2  &  (1, shafter, 15) &  (1, shafter, 15) &  3.353189 &  5.572850 &   0.496553 &  0.169842 \\
6  &  (1, shafter, 18) &  (1, shafter, 18) &  2.309084 &  4.330305 &   0.413509 &  0.131661 \\
9  &  (2, donovan, 15) &  (2, donovan, 15) &  6.698192 &  7.547663 &   0.630592 &  0.175792 \\
12 &  (2, donovan, 18) &  (2, donovan, 18) &  6.324331 &  6.607726 &   0.609086 &  0.154963 \\
4  &  (2, donovan, 20) &  (2, donovan, 20) &  6.775750 &  7.562647 &   0.652561 &  0.177358 \\
3  &  (2, elcajon, 17) &  (2, elcajon, 17) &  3.379753 &  7.288420 &   0.300214 &  0.191092 \\
7  &  (2, elcajon, 19) &  (2, elcajon, 19) &  4.038051 &  7.618134 &   0.359570 &  0.199615 \\
14 &  (2, elcajon, 21) &  (2, elcajon, 21) &  3.970448 &  7.246587 &   0.353550 &  0.189880 \\
0  &  (3, donovan, 11) &  (3, donovan, 11) &  5.146091 &  5.509514 &   0.433960 &  0.184209 \\
1  &  (3, donovan, 12) &  (3, donovan, 12) &  3.858934 &  6.274502 &   0.325327 &  0.209866 \\
17 &  (3, donovan, 13) &  (3, donovan, 13) &  6.397274 &  8.225554 &   0.544893 &  0.276492 \\
8  &  (3, elcajon, 15) &  (3, elcajon, 15) &  3.956452 &  5.724768 &   0.249405 &  0.253835 \\
16 &  (3, elcajon, 18) &  (3, elcajon, 18) &  3.635693 &  4.002478 &   0.227972 &  0.177812 \\
15 &  (3, elcajon, 20) &  (3, elcajon, 20) &  4.175166 &  5.877268 &   0.261707 &  0.261232 \\
\bottomrule
\end{tabular}

\caption{Level 0 train results for linear regression}
\end{table}
\begin{table}[H]
\centering
\scriptsize
\begin{tabular}{lllrrrr}
\toprule
{} &             Model &  Testing Location &   NO2 MAE &    O3 MAE &  NO2 CvMAE &  O3 CvMAE \\
\midrule
16 &  (1, donovan, 19) &  (1, donovan, 19) &  2.850930 &  3.710558 &   0.673331 &  0.097286 \\
5  &  (1, donovan, 21) &  (1, donovan, 21) &  2.543194 &  4.096572 &   0.600650 &  0.107407 \\
8  &  (1, elcajon, 11) &  (1, elcajon, 11) &  2.053270 &  5.128018 &   0.378209 &  0.135625 \\
2  &  (1, elcajon, 12) &  (1, elcajon, 12) &  2.181817 &  5.183898 &   0.401887 &  0.137103 \\
9  &  (1, elcajon, 13) &  (1, elcajon, 13) &  2.072161 &  4.386177 &   0.381688 &  0.116005 \\
0  &  (1, shafter, 15) &  (1, shafter, 15) &  3.386877 &  5.589703 &   0.500763 &  0.171725 \\
6  &  (1, shafter, 18) &  (1, shafter, 18) &  2.313071 &  4.412545 &   0.419196 &  0.134887 \\
14 &  (2, donovan, 15) &  (2, donovan, 15) &  6.652718 &  7.528181 &   0.624712 &  0.175110 \\
22 &  (2, donovan, 18) &  (2, donovan, 18) &  6.385480 &  6.706138 &   0.605916 &  0.158235 \\
4  &  (2, donovan, 20) &  (2, donovan, 20) &  6.803972 &  7.533899 &   0.645627 &  0.177766 \\
13 &  (2, elcajon, 17) &  (2, elcajon, 17) &  3.350610 &  7.275296 &   0.297892 &  0.187986 \\
1  &  (2, elcajon, 19) &  (2, elcajon, 19) &  4.136211 &  7.747041 &   0.363729 &  0.200693 \\
3  &  (2, elcajon, 21) &  (2, elcajon, 21) &  4.037583 &  7.344045 &   0.355056 &  0.190253 \\
20 &  (2, shafter, 11) &  (2, shafter, 11) &  4.416675 &  7.262233 &   0.325816 &  0.230545 \\
21 &  (2, shafter, 12) &  (2, shafter, 12) &  4.619718 &  6.987870 &   0.344165 &  0.221812 \\
11 &  (2, shafter, 13) &  (2, shafter, 13) &  4.576023 &  7.411089 &   0.339026 &  0.233867 \\
23 &  (3, donovan, 11) &  (3, donovan, 11) &  5.193754 &  5.600240 &   0.431705 &  0.188033 \\
15 &  (3, donovan, 12) &  (3, donovan, 12) &  3.967916 &  6.324312 &   0.330162 &  0.212043 \\
12 &  (3, donovan, 13) &  (3, donovan, 13) &  6.487443 &  8.338580 &   0.544654 &  0.280263 \\
24 &  (3, elcajon, 15) &  (3, elcajon, 15) &  3.975396 &  5.762910 &   0.249059 &  0.256456 \\
17 &  (3, elcajon, 18) &  (3, elcajon, 18) &  3.618518 &  4.013838 &   0.226705 &  0.178048 \\
10 &  (3, elcajon, 20) &  (3, elcajon, 20) &  4.169921 &  5.865262 &   0.261620 &  0.259674 \\
18 &  (3, shafter, 17) &  (3, shafter, 17) &  5.385507 &  6.088492 &   0.393443 &  0.269503 \\
7  &  (3, shafter, 19) &  (3, shafter, 19) &  5.993148 &  6.998718 &   0.437834 &  0.309794 \\
19 &  (3, shafter, 21) &  (3, shafter, 21) &  4.812474 &  5.096993 &   0.351579 &  0.225615 \\
\bottomrule
\end{tabular}

\caption{Level 0 test results for linear regression}
\end{table}

\begin{table}[H]
\centering
\scriptsize
\begin{tabular}{lllrrrr}
\toprule
{} &             Model & Testing Location &   NO2 MAE &    O3 MAE &  NO2 CvMAE &  O3 CvMAE \\
\midrule
20 &  (1, donovan, 19) &     (2, elcajon) &  2.850930 &  3.710558 &   0.673331 &  0.097286 \\
21 &  (1, donovan, 19) &     (3, shafter) &  2.850930 &  3.710558 &   0.673331 &  0.097286 \\
24 &  (1, donovan, 21) &     (2, elcajon) &  2.543194 &  4.096572 &   0.600650 &  0.107407 \\
25 &  (1, donovan, 21) &     (3, shafter) &  2.543194 &  4.096572 &   0.600650 &  0.107407 \\
18 &  (1, elcajon, 11) &     (2, shafter) &  2.053270 &  5.128018 &   0.378209 &  0.135625 \\
19 &  (1, elcajon, 11) &     (3, donovan) &  2.053270 &  5.128018 &   0.378209 &  0.135625 \\
33 &  (1, elcajon, 12) &     (2, shafter) &  2.181817 &  5.183898 &   0.401887 &  0.137103 \\
34 &  (1, elcajon, 12) &     (3, donovan) &  2.181817 &  5.183898 &   0.401887 &  0.137103 \\
8  &  (1, elcajon, 13) &     (2, shafter) &  2.072161 &  4.386177 &   0.381688 &  0.116005 \\
9  &  (1, elcajon, 13) &     (3, donovan) &  2.072161 &  4.386177 &   0.381688 &  0.116005 \\
4  &  (1, shafter, 15) &     (2, donovan) &  3.386877 &  5.589703 &   0.500763 &  0.171725 \\
5  &  (1, shafter, 15) &     (3, elcajon) &  3.386877 &  5.589703 &   0.500763 &  0.171725 \\
10 &  (1, shafter, 18) &     (2, donovan) &  2.313071 &  4.412545 &   0.419196 &  0.134887 \\
11 &  (1, shafter, 18) &     (3, elcajon) &  2.313071 &  4.412545 &   0.419196 &  0.134887 \\
16 &  (2, donovan, 15) &     (1, shafter) &  6.652718 &  7.528181 &   0.624712 &  0.175110 \\
17 &  (2, donovan, 15) &     (3, elcajon) &  6.652718 &  7.528181 &   0.624712 &  0.175110 \\
22 &  (2, donovan, 18) &     (1, shafter) &  6.385480 &  6.706138 &   0.605916 &  0.158235 \\
23 &  (2, donovan, 18) &     (3, elcajon) &  6.385480 &  6.706138 &   0.605916 &  0.158235 \\
7  &  (2, donovan, 20) &     (3, elcajon) &  6.803972 &  7.533899 &   0.645627 &  0.177766 \\
6  &  (2, elcajon, 17) &     (3, shafter) &  3.350610 &  7.275296 &   0.297892 &  0.187986 \\
12 &  (2, elcajon, 19) &     (1, donovan) &  4.136211 &  7.747041 &   0.363729 &  0.200693 \\
13 &  (2, elcajon, 19) &     (3, shafter) &  4.136211 &  7.747041 &   0.363729 &  0.200693 \\
26 &  (2, elcajon, 21) &     (1, donovan) &  4.037583 &  7.344045 &   0.355056 &  0.190253 \\
27 &  (2, elcajon, 21) &     (3, shafter) &  4.037583 &  7.344045 &   0.355056 &  0.190253 \\
0  &  (3, donovan, 11) &     (1, elcajon) &  5.193754 &  5.600240 &   0.431705 &  0.188033 \\
1  &  (3, donovan, 11) &     (2, shafter) &  5.193754 &  5.600240 &   0.431705 &  0.188033 \\
2  &  (3, donovan, 12) &     (1, elcajon) &  3.967916 &  6.324312 &   0.330162 &  0.212043 \\
3  &  (3, donovan, 12) &     (2, shafter) &  3.967916 &  6.324312 &   0.330162 &  0.212043 \\
31 &  (3, donovan, 13) &     (1, elcajon) &  6.487443 &  8.338580 &   0.544654 &  0.280263 \\
32 &  (3, donovan, 13) &     (2, shafter) &  6.487443 &  8.338580 &   0.544654 &  0.280263 \\
14 &  (3, elcajon, 15) &     (1, shafter) &  3.975396 &  5.762910 &   0.249059 &  0.256456 \\
15 &  (3, elcajon, 15) &     (2, donovan) &  3.975396 &  5.762910 &   0.249059 &  0.256456 \\
29 &  (3, elcajon, 18) &     (1, shafter) &  3.618518 &  4.013838 &   0.226705 &  0.178048 \\
30 &  (3, elcajon, 18) &     (2, donovan) &  3.618518 &  4.013838 &   0.226705 &  0.178048 \\
28 &  (3, elcajon, 20) &     (2, donovan) &  4.169921 &  5.865262 &   0.261620 &  0.259674 \\
\bottomrule
\end{tabular}

\caption{Level 1 train results for linear regression}
\end{table}
\begin{table}[H]
\centering
\scriptsize
\begin{tabular}{lllrrrr}
\toprule
{} &             Model & Testing Location &    NO2 MAE &     O3 MAE &  NO2 CvMAE &  O3 CvMAE \\
\midrule
29 &  (1, donovan, 19) &     (2, elcajon) &   4.721054 &  13.743547 &   0.415159 &  0.356037 \\
30 &  (1, donovan, 19) &     (3, shafter) &   9.750380 &  44.837383 &   0.712322 &  1.984698 \\
9  &  (1, donovan, 21) &     (2, elcajon) &   5.435679 &  11.265806 &   0.478001 &  0.291849 \\
10 &  (1, donovan, 21) &     (3, shafter) &   8.163074 &  24.216401 &   0.596360 &  1.071923 \\
15 &  (1, elcajon, 11) &     (2, shafter) &   5.859866 &  23.973826 &   0.432280 &  0.761068 \\
16 &  (1, elcajon, 11) &     (3, donovan) &   6.238380 &  11.191060 &   0.518534 &  0.375750 \\
4  &  (1, elcajon, 12) &     (2, shafter) &   6.504934 &  17.526472 &   0.484612 &  0.556333 \\
5  &  (1, elcajon, 12) &     (3, donovan) &   6.446498 &  10.332413 &   0.536399 &  0.346427 \\
17 &  (1, elcajon, 13) &     (2, shafter) &   6.987382 &  29.507606 &   0.517678 &  0.931151 \\
18 &  (1, elcajon, 13) &     (3, donovan) &  10.459595 &  38.840301 &   0.878137 &  1.305440 \\
0  &  (1, shafter, 15) &     (2, donovan) &   6.879212 &  18.603261 &   0.645980 &  0.432722 \\
1  &  (1, shafter, 15) &     (3, elcajon) &   6.870887 &   7.834289 &   0.430462 &  0.348635 \\
11 &  (1, shafter, 18) &     (2, donovan) &   6.108138 &  32.634696 &   0.579599 &  0.770032 \\
12 &  (1, shafter, 18) &     (3, elcajon) &   5.490944 &   9.172539 &   0.344015 &  0.406881 \\
25 &  (2, donovan, 15) &     (1, shafter) &   9.377802 &   8.475076 &   1.386546 &  0.260368 \\
26 &  (2, donovan, 15) &     (3, elcajon) &   6.425870 &   7.199287 &   0.402582 &  0.320377 \\
40 &  (2, donovan, 18) &     (1, shafter) &   8.651816 &   8.316647 &   1.567962 &  0.254231 \\
41 &  (2, donovan, 18) &     (3, elcajon) &   6.229479 &   5.859996 &   0.390285 &  0.259941 \\
8  &  (2, donovan, 20) &     (3, elcajon) &   6.052090 &   7.586682 &   0.379707 &  0.335886 \\
24 &  (2, elcajon, 17) &     (3, shafter) &  12.827964 &  17.886811 &   0.937158 &  0.791748 \\
2  &  (2, elcajon, 19) &     (1, donovan) &   4.800626 &   4.762190 &   1.133809 &  0.124859 \\
3  &  (2, elcajon, 19) &     (3, shafter) &  14.483825 &  21.333860 &   1.058128 &  0.944329 \\
6  &  (2, elcajon, 21) &     (1, donovan) &   3.583996 &   4.596593 &   0.846466 &  0.120517 \\
7  &  (2, elcajon, 21) &     (3, shafter) &  12.789354 &   9.587360 &   0.934337 &  0.424378 \\
36 &  (2, shafter, 11) &     (1, elcajon) &   7.210789 &   7.536208 &   1.328214 &  0.199317 \\
37 &  (2, shafter, 11) &     (3, donovan) &   6.656738 &   9.714873 &   0.553308 &  0.326186 \\
38 &  (2, shafter, 12) &     (1, elcajon) &   6.664027 &   7.139145 &   1.227502 &  0.188815 \\
39 &  (2, shafter, 12) &     (3, donovan) &   6.732293 &   9.466997 &   0.560180 &  0.317411 \\
20 &  (2, shafter, 13) &     (1, elcajon) &   9.198244 &   9.330914 &   1.694300 &  0.246783 \\
21 &  (2, shafter, 13) &     (3, donovan) &   9.655395 &  14.511711 &   0.810620 &  0.487745 \\
42 &  (3, donovan, 11) &     (1, elcajon) &   3.692888 &   7.873262 &   0.680223 &  0.208231 \\
43 &  (3, donovan, 11) &     (2, shafter) &   9.106516 &   7.761055 &   0.671783 &  0.246381 \\
27 &  (3, donovan, 12) &     (1, elcajon) &   7.332869 &   8.791024 &   1.350701 &  0.232504 \\
28 &  (3, donovan, 12) &     (2, shafter) &   9.222817 &   9.973596 &   0.687092 &  0.316586 \\
22 &  (3, donovan, 13) &     (1, elcajon) &   4.669914 &   8.973694 &   0.860190 &  0.237335 \\
23 &  (3, donovan, 13) &     (2, shafter) &   5.636151 &  12.978008 &   0.417569 &  0.409538 \\
44 &  (3, elcajon, 15) &     (1, shafter) &   4.827194 &   8.805387 &   0.713720 &  0.270516 \\
45 &  (3, elcajon, 15) &     (2, donovan) &   6.621335 &   8.375271 &   0.621765 &  0.194813 \\
31 &  (3, elcajon, 18) &     (1, shafter) &   4.991286 &   6.798044 &   0.904567 &  0.207809 \\
32 &  (3, elcajon, 18) &     (2, donovan) &   7.125273 &   7.871634 &   0.676115 &  0.185735 \\
19 &  (3, elcajon, 20) &     (2, donovan) &   8.969028 &   8.574747 &   0.851068 &  0.202325 \\
33 &  (3, shafter, 17) &     (2, elcajon) &   8.463228 &  13.245548 &   0.752440 &  0.342250 \\
13 &  (3, shafter, 19) &     (1, donovan) &   7.840513 &  24.786326 &   1.851768 &  0.649866 \\
14 &  (3, shafter, 19) &     (2, elcajon) &   6.344760 &  10.436176 &   0.557944 &  0.270357 \\
34 &  (3, shafter, 21) &     (1, donovan) &   3.141038 &   8.195038 &   0.741848 &  0.214863 \\
35 &  (3, shafter, 21) &     (2, elcajon) &   8.051262 &   8.951980 &   0.708009 &  0.231908 \\
\bottomrule
\end{tabular}

\caption{Level 1 test results for linear regression}
\end{table}

\begin{table}[H]
\centering
\scriptsize
\begin{tabular}{llrrrr}
\toprule
{} &                               Model &   NO2 MAE &    O3 MAE &  NO2 CvMAE &  O3 CvMAE \\
\midrule
0  &  (13, \{(3, donovan), (2, shafter)\}) &  6.355285 &  9.091896 &   0.511315 &  0.299221 \\
1  &  (18, \{(1, shafter), (2, donovan)\}) &  5.871708 &  6.760317 &   0.616786 &  0.167255 \\
2  &  (12, \{(1, elcajon), (3, donovan)\}) &  4.127693 &  6.698369 &   0.384564 &  0.213444 \\
3  &  (19, \{(3, shafter), (1, donovan)\}) &  5.569429 &  7.871177 &   0.463862 &  0.310413 \\
4  &  (15, \{(2, donovan), (3, elcajon)\}) &  4.952119 &  6.508162 &   0.340535 &  0.232814 \\
5  &  (15, \{(1, shafter), (2, donovan)\}) &  5.273225 &  7.551699 &   0.599360 &  0.198638 \\
6  &  (19, \{(3, shafter), (2, elcajon)\}) &  5.982334 &  7.782698 &   0.457800 &  0.289529 \\
7  &  (13, \{(1, elcajon), (2, shafter)\}) &  4.597032 &  7.339406 &   0.432687 &  0.216700 \\
8  &  (11, \{(1, elcajon), (2, shafter)\}) &  4.064288 &  6.914293 &   0.381481 &  0.204848 \\
9  &  (11, \{(3, donovan), (2, shafter)\}) &  5.395918 &  6.412672 &   0.431885 &  0.211607 \\
10 &  (12, \{(3, donovan), (2, shafter)\}) &  5.010363 &  7.233493 &   0.402190 &  0.238330 \\
11 &  (21, \{(1, donovan), (2, elcajon)\}) &  3.746372 &  6.286799 &   0.429642 &  0.163590 \\
12 &  (19, \{(1, donovan), (2, elcajon)\}) &  3.874982 &  6.455872 &   0.444391 &  0.167989 \\
13 &  (15, \{(1, shafter), (3, elcajon)\}) &  3.995235 &  6.143927 &   0.292257 &  0.245933 \\
14 &  (11, \{(1, elcajon), (3, donovan)\}) &  4.764080 &  5.866409 &   0.443432 &  0.187137 \\
15 &  (12, \{(1, elcajon), (2, shafter)\}) &  4.486508 &  6.936752 &   0.421359 &  0.205881 \\
16 &  (13, \{(1, elcajon), (3, donovan)\}) &  5.755871 &  8.116783 &   0.546095 &  0.258029 \\
17 &  (21, \{(3, shafter), (2, elcajon)\}) &  5.272058 &  6.103408 &   0.403446 &  0.227056 \\
18 &  (21, \{(3, shafter), (1, donovan)\}) &  4.537362 &  5.222198 &   0.377904 &  0.205946 \\
19 &  (18, \{(1, shafter), (3, elcajon)\}) &  3.686667 &  4.247455 &   0.244767 &  0.181365 \\
20 &  (18, \{(2, donovan), (3, elcajon)\}) &  4.559396 &  5.117607 &   0.314524 &  0.183405 \\
\bottomrule
\end{tabular}

\caption{Level 2 train results for linear regression}
\end{table}
\begin{table}[H]
\centering
\scriptsize
\begin{tabular}{llrrrr}
\toprule
{} &                               Model &   NO2 MAE &     O3 MAE &  NO2 CvMAE &  O3 CvMAE \\
\midrule
0  &  (15, \{(1, shafter), (3, elcajon)\}) &  6.481640 &  11.151591 &   0.608647 &  0.259392 \\
1  &  (18, \{(1, shafter), (3, elcajon)\}) &  6.873614 &   7.939836 &   0.652235 &  0.187344 \\
2  &  (15, \{(3, elcajon), (2, donovan)\}) &  5.694266 &   9.038393 &   0.841920 &  0.277674 \\
3  &  (15, \{(3, elcajon), (2, donovan)\}) &  5.694266 &   9.038393 &   0.841920 &  0.277674 \\
4  &  (15, \{(1, shafter), (2, donovan)\}) &  5.834374 &   7.015913 &   0.365524 &  0.312216 \\
5  &  (18, \{(1, shafter), (3, elcajon)\}) &  6.873614 &   7.939836 &   0.652235 &  0.187344 \\
6  &  (15, \{(1, shafter), (2, donovan)\}) &  5.834374 &   7.015913 &   0.365524 &  0.312216 \\
7  &  (18, \{(3, elcajon), (2, donovan)\}) &  4.799088 &   6.939131 &   0.869735 &  0.212122 \\
8  &  (15, \{(1, shafter), (3, elcajon)\}) &  6.481640 &  11.151591 &   0.608647 &  0.259392 \\
9  &  (18, \{(1, shafter), (2, donovan)\}) &  4.802131 &   6.440659 &   0.300860 &  0.285699 \\
10 &  (18, \{(1, shafter), (2, donovan)\}) &  4.802131 &   6.440659 &   0.300860 &  0.285699 \\
11 &  (18, \{(3, elcajon), (2, donovan)\}) &  4.799088 &   6.939131 &   0.869735 &  0.212122 \\
\bottomrule
\end{tabular}

\caption{Level 2 test results for linear regression}
\end{table}

\begin{table}[H]
\centering
\scriptsize
\begin{tabular}{lllrrrr}
\toprule
{} & Model &          Test &   NO2 MAE &    O3 MAE &  NO2 CvMAE &  O3 CvMAE \\
\midrule
0  &    11 &  (1, elcajon) &  4.697781 &  6.444363 &   0.876128 &  0.170259 \\
1  &    11 &  (3, donovan) &  5.522252 &  6.028935 &   0.465681 &  0.201576 \\
2  &    11 &  (2, shafter) &  4.963687 &  7.589506 &   0.369731 &  0.240540 \\
3  &    17 &  (3, shafter) &  5.382709 &  6.816198 &   0.393243 &  0.298917 \\
4  &    17 &  (2, elcajon) &  4.467562 &  8.433621 &   0.396841 &  0.221118 \\
5  &    18 &  (1, shafter) &  4.306174 &  6.852810 &   0.771146 &  0.208356 \\
6  &    18 &  (2, donovan) &  6.348767 &  7.071557 &   0.611439 &  0.165841 \\
7  &    18 &  (3, elcajon) &  3.827298 &  4.371915 &   0.239986 &  0.194224 \\
8  &    13 &  (1, elcajon) &  4.466222 &  6.222771 &   0.832942 &  0.164404 \\
9  &    13 &  (3, donovan) &  6.928068 &  9.174213 &   0.590104 &  0.308380 \\
10 &    13 &  (2, shafter) &  4.700801 &  8.639669 &   0.346643 &  0.275086 \\
11 &    12 &  (1, elcajon) &  4.807731 &  6.856014 &   0.896633 &  0.181135 \\
12 &    12 &  (3, donovan) &  4.931128 &  7.150968 &   0.415718 &  0.239182 \\
13 &    12 &  (2, shafter) &  5.003767 &  8.154915 &   0.372253 &  0.257888 \\
14 &    20 &  (2, donovan) &  6.927276 &  7.646314 &   0.667154 &  0.179320 \\
15 &    20 &  (3, elcajon) &  4.535472 &  6.149316 &   0.284291 &  0.273324 \\
16 &    15 &  (1, shafter) &  4.202212 &  7.293891 &   0.622279 &  0.222294 \\
17 &    15 &  (2, donovan) &  6.447277 &  7.991731 &   0.606970 &  0.186135 \\
18 &    15 &  (3, elcajon) &  4.356170 &  6.147377 &   0.274603 &  0.272574 \\
19 &    19 &  (3, shafter) &  6.319686 &  8.076446 &   0.461695 &  0.354184 \\
20 &    19 &  (1, donovan) &  4.330704 &  9.153789 &   1.028971 &  0.241691 \\
21 &    19 &  (2, elcajon) &  5.037780 &  9.303649 &   0.448591 &  0.243780 \\
22 &    21 &  (3, shafter) &  5.365747 &  5.566882 &   0.392004 &  0.244130 \\
23 &    21 &  (1, donovan) &  3.230923 &  4.987529 &   0.767664 &  0.131688 \\
24 &    21 &  (2, elcajon) &  4.971744 &  7.792549 &   0.442711 &  0.204185 \\
\bottomrule
\end{tabular}

\caption{Level 3 train results for linear regression}
\end{table}
\begin{table}[H]
\centering
\scriptsize
\begin{tabular}{lrrrrr}
\toprule
{} &  Model &   NO2 MAE &    O3 MAE &  NO2 CvMAE &  O3 CvMAE \\
\midrule
0 &   20.0 &  4.971668 &  6.454269 &   0.342330 &  0.230975 \\
1 &   21.0 &  3.690337 &  6.224828 &   0.424093 &  0.163753 \\
2 &   18.0 &  4.741748 &  5.374133 &   0.337748 &  0.191508 \\
3 &   19.0 &  3.811553 &  6.384892 &   0.438023 &  0.167964 \\
4 &   15.0 &  5.293941 &  7.249578 &   0.401235 &  0.250439 \\
5 &   17.0 &  3.350610 &  7.275296 &   0.297892 &  0.187986 \\
6 &   11.0 &  4.674364 &  5.814669 &   0.432580 &  0.185817 \\
7 &   12.0 &  4.197401 &  6.607890 &   0.391736 &  0.210993 \\
8 &   13.0 &  5.625144 &  8.039349 &   0.533456 &  0.256374 \\
\bottomrule
\end{tabular}

\caption{Level 3 test results for linear regression}
\end{table}

\subsection{Benchmarks for NN[2]}
\label{sec:results-nn2}

\begin{table}[H]
\centering
\scriptsize
\begin{tabular}{lllrrrr}
\toprule
{} &             Model &  Testing Location &   NO2 MAE &    O3 MAE &  NO2 CvMAE &  O3 CvMAE \\
\midrule
16 &  (1, donovan, 19) &  (1, donovan, 19) &  2.376601 &  4.059605 &   0.564678 &  0.107187 \\
5  &  (1, donovan, 21) &  (1, donovan, 21) &  2.106573 &  3.643729 &   0.500520 &  0.096207 \\
8  &  (1, elcajon, 11) &  (1, elcajon, 11) &  1.363333 &  4.308309 &   0.254259 &  0.113825 \\
2  &  (1, elcajon, 12) &  (1, elcajon, 12) &  1.731689 &  3.433356 &   0.322957 &  0.090709 \\
9  &  (1, elcajon, 13) &  (1, elcajon, 13) &  1.592846 &  3.237433 &   0.297063 &  0.085532 \\
0  &  (1, shafter, 15) &  (1, shafter, 15) &  2.197847 &  3.945468 &   0.325466 &  0.120245 \\
6  &  (1, shafter, 18) &  (1, shafter, 18) &  2.200961 &  4.377567 &   0.394146 &  0.133098 \\
14 &  (2, donovan, 15) &  (2, donovan, 15) &  5.774000 &  6.602082 &   0.543585 &  0.153768 \\
22 &  (2, donovan, 18) &  (2, donovan, 18) &  5.147102 &  6.150489 &   0.495709 &  0.144240 \\
4  &  (2, donovan, 20) &  (2, donovan, 20) &  5.239533 &  6.653606 &   0.504610 &  0.156039 \\
13 &  (2, elcajon, 17) &  (2, elcajon, 17) &  2.531008 &  5.177073 &   0.224822 &  0.135736 \\
1  &  (2, elcajon, 19) &  (2, elcajon, 19) &  3.021720 &  6.658900 &   0.269070 &  0.174481 \\
3  &  (2, elcajon, 21) &  (2, elcajon, 21) &  2.697276 &  5.373738 &   0.240180 &  0.140806 \\
20 &  (2, shafter, 11) &  (2, shafter, 11) &  3.871853 &  5.892196 &   0.288404 &  0.186746 \\
21 &  (2, shafter, 12) &  (2, shafter, 12) &  4.011850 &  6.202820 &   0.298460 &  0.196155 \\
11 &  (2, shafter, 13) &  (2, shafter, 13) &  3.847771 &  6.404214 &   0.283740 &  0.203909 \\
23 &  (3, donovan, 11) &  (3, donovan, 11) &  3.364954 &  4.612540 &   0.283760 &  0.154219 \\
15 &  (3, donovan, 12) &  (3, donovan, 12) &  3.123734 &  4.774751 &   0.263346 &  0.159704 \\
12 &  (3, donovan, 13) &  (3, donovan, 13) &  3.544877 &  5.484555 &   0.301938 &  0.184356 \\
24 &  (3, elcajon, 15) &  (3, elcajon, 15) &  2.328371 &  4.167490 &   0.146775 &  0.184786 \\
17 &  (3, elcajon, 18) &  (3, elcajon, 18) &  2.520757 &  3.234963 &   0.158061 &  0.143715 \\
10 &  (3, elcajon, 20) &  (3, elcajon, 20) &  2.357596 &  3.846291 &   0.147778 &  0.170959 \\
18 &  (3, shafter, 17) &  (3, shafter, 17) &  3.481401 &  4.433496 &   0.254340 &  0.194426 \\
7  &  (3, shafter, 19) &  (3, shafter, 19) &  3.921369 &  4.727419 &   0.286482 &  0.207316 \\
19 &  (3, shafter, 21) &  (3, shafter, 21) &  3.053778 &  4.597083 &   0.223099 &  0.201600 \\
\bottomrule
\end{tabular}

\caption{Level 0 train results for NN[2]}
\end{table}
\begin{table}[H]
\centering
\scriptsize
\begin{tabular}{lllrrrr}
\toprule
{} &             Model &  Testing Location &   NO2 MAE &    O3 MAE &  NO2 CvMAE &  O3 CvMAE \\
\midrule
16 &  (1, donovan, 19) &  (1, donovan, 19) &  2.406302 &  4.068393 &   0.568319 &  0.106668 \\
5  &  (1, donovan, 21) &  (1, donovan, 21) &  2.121591 &  3.700573 &   0.501076 &  0.097024 \\
8  &  (1, elcajon, 11) &  (1, elcajon, 11) &  1.372707 &  4.321155 &   0.252850 &  0.114286 \\
2  &  (1, elcajon, 12) &  (1, elcajon, 12) &  1.751919 &  3.411224 &   0.322700 &  0.090220 \\
9  &  (1, elcajon, 13) &  (1, elcajon, 13) &  1.587393 &  3.206342 &   0.292395 &  0.084801 \\
0  &  (1, shafter, 15) &  (1, shafter, 15) &  2.201070 &  3.989497 &   0.325437 &  0.122564 \\
6  &  (1, shafter, 18) &  (1, shafter, 18) &  2.302697 &  4.502786 &   0.417316 &  0.137645 \\
14 &  (2, donovan, 15) &  (2, donovan, 15) &  5.782659 &  6.622596 &   0.543010 &  0.154045 \\
22 &  (2, donovan, 18) &  (2, donovan, 18) &  5.275075 &  6.264368 &   0.500550 &  0.147811 \\
4  &  (2, donovan, 20) &  (2, donovan, 20) &  5.336130 &  6.772800 &   0.506344 &  0.159808 \\
13 &  (2, elcajon, 17) &  (2, elcajon, 17) &  2.556621 &  5.126892 &   0.227301 &  0.132473 \\
1  &  (2, elcajon, 19) &  (2, elcajon, 19) &  3.082288 &  6.774753 &   0.271049 &  0.175505 \\
3  &  (2, elcajon, 21) &  (2, elcajon, 21) &  2.762940 &  5.463632 &   0.242967 &  0.141540 \\
20 &  (2, shafter, 11) &  (2, shafter, 11) &  3.874197 &  5.946993 &   0.285798 &  0.188792 \\
21 &  (2, shafter, 12) &  (2, shafter, 12) &  4.048163 &  6.254481 &   0.301585 &  0.198533 \\
11 &  (2, shafter, 13) &  (2, shafter, 13) &  3.882250 &  6.439484 &   0.287627 &  0.203206 \\
23 &  (3, donovan, 11) &  (3, donovan, 11) &  3.447279 &  4.675427 &   0.286538 &  0.156982 \\
15 &  (3, donovan, 12) &  (3, donovan, 12) &  3.205964 &  4.806858 &   0.266761 &  0.161165 \\
12 &  (3, donovan, 13) &  (3, donovan, 13) &  3.665248 &  5.595738 &   0.307716 &  0.188075 \\
24 &  (3, elcajon, 15) &  (3, elcajon, 15) &  2.317087 &  4.174768 &   0.145166 &  0.185782 \\
17 &  (3, elcajon, 18) &  (3, elcajon, 18) &  2.540051 &  3.222305 &   0.159138 &  0.142937 \\
10 &  (3, elcajon, 20) &  (3, elcajon, 20) &  2.365656 &  3.836465 &   0.148421 &  0.169852 \\
18 &  (3, shafter, 17) &  (3, shafter, 17) &  3.498404 &  4.454751 &   0.255579 &  0.197187 \\
7  &  (3, shafter, 19) &  (3, shafter, 19) &  3.902972 &  4.688188 &   0.285135 &  0.207520 \\
19 &  (3, shafter, 21) &  (3, shafter, 21) &  3.050720 &  4.577955 &   0.222873 &  0.202640 \\
\bottomrule
\end{tabular}

\caption{Level 0 test results for NN[2]}
\end{table}

\begin{table}[H]
\centering
\scriptsize
\begin{tabular}{lllrrrr}
\toprule
{} &             Model & Testing Location &   NO2 MAE &    O3 MAE &  NO2 CvMAE &  O3 CvMAE \\
\midrule
29 &  (1, donovan, 19) &     (2, elcajon) &  2.406302 &  4.068393 &   0.568319 &  0.106668 \\
30 &  (1, donovan, 19) &     (3, shafter) &  2.406302 &  4.068393 &   0.568319 &  0.106668 \\
9  &  (1, donovan, 21) &     (2, elcajon) &  2.121591 &  3.700573 &   0.501076 &  0.097024 \\
10 &  (1, donovan, 21) &     (3, shafter) &  2.121591 &  3.700573 &   0.501076 &  0.097024 \\
15 &  (1, elcajon, 11) &     (2, shafter) &  1.372707 &  4.321155 &   0.252850 &  0.114286 \\
16 &  (1, elcajon, 11) &     (3, donovan) &  1.372707 &  4.321155 &   0.252850 &  0.114286 \\
4  &  (1, elcajon, 12) &     (2, shafter) &  1.751919 &  3.411224 &   0.322700 &  0.090220 \\
5  &  (1, elcajon, 12) &     (3, donovan) &  1.751919 &  3.411224 &   0.322700 &  0.090220 \\
17 &  (1, elcajon, 13) &     (2, shafter) &  1.587393 &  3.206342 &   0.292395 &  0.084801 \\
18 &  (1, elcajon, 13) &     (3, donovan) &  1.587393 &  3.206342 &   0.292395 &  0.084801 \\
0  &  (1, shafter, 15) &     (2, donovan) &  2.201070 &  3.989497 &   0.325437 &  0.122564 \\
1  &  (1, shafter, 15) &     (3, elcajon) &  2.201070 &  3.989497 &   0.325437 &  0.122564 \\
11 &  (1, shafter, 18) &     (2, donovan) &  2.302697 &  4.502786 &   0.417316 &  0.137645 \\
12 &  (1, shafter, 18) &     (3, elcajon) &  2.302697 &  4.502786 &   0.417316 &  0.137645 \\
25 &  (2, donovan, 15) &     (1, shafter) &  5.782659 &  6.622596 &   0.543010 &  0.154045 \\
26 &  (2, donovan, 15) &     (3, elcajon) &  5.782659 &  6.622596 &   0.543010 &  0.154045 \\
40 &  (2, donovan, 18) &     (1, shafter) &  5.275075 &  6.264368 &   0.500550 &  0.147811 \\
41 &  (2, donovan, 18) &     (3, elcajon) &  5.275075 &  6.264368 &   0.500550 &  0.147811 \\
8  &  (2, donovan, 20) &     (3, elcajon) &  5.336130 &  6.772800 &   0.506344 &  0.159808 \\
24 &  (2, elcajon, 17) &     (3, shafter) &  2.556621 &  5.126892 &   0.227301 &  0.132473 \\
2  &  (2, elcajon, 19) &     (1, donovan) &  3.082288 &  6.774753 &   0.271049 &  0.175505 \\
3  &  (2, elcajon, 19) &     (3, shafter) &  3.082288 &  6.774753 &   0.271049 &  0.175505 \\
6  &  (2, elcajon, 21) &     (1, donovan) &  2.762940 &  5.463632 &   0.242967 &  0.141540 \\
7  &  (2, elcajon, 21) &     (3, shafter) &  2.762940 &  5.463632 &   0.242967 &  0.141540 \\
36 &  (2, shafter, 11) &     (1, elcajon) &  3.874197 &  5.946993 &   0.285798 &  0.188792 \\
37 &  (2, shafter, 11) &     (3, donovan) &  3.874197 &  5.946993 &   0.285798 &  0.188792 \\
38 &  (2, shafter, 12) &     (1, elcajon) &  4.048163 &  6.254481 &   0.301585 &  0.198533 \\
39 &  (2, shafter, 12) &     (3, donovan) &  4.048163 &  6.254481 &   0.301585 &  0.198533 \\
20 &  (2, shafter, 13) &     (1, elcajon) &  3.882250 &  6.439484 &   0.287627 &  0.203206 \\
21 &  (2, shafter, 13) &     (3, donovan) &  3.882250 &  6.439484 &   0.287627 &  0.203206 \\
42 &  (3, donovan, 11) &     (1, elcajon) &  3.447279 &  4.675427 &   0.286538 &  0.156982 \\
43 &  (3, donovan, 11) &     (2, shafter) &  3.447279 &  4.675427 &   0.286538 &  0.156982 \\
27 &  (3, donovan, 12) &     (1, elcajon) &  3.205964 &  4.806858 &   0.266761 &  0.161165 \\
28 &  (3, donovan, 12) &     (2, shafter) &  3.205964 &  4.806858 &   0.266761 &  0.161165 \\
22 &  (3, donovan, 13) &     (1, elcajon) &  3.665248 &  5.595738 &   0.307716 &  0.188075 \\
23 &  (3, donovan, 13) &     (2, shafter) &  3.665248 &  5.595738 &   0.307716 &  0.188075 \\
44 &  (3, elcajon, 15) &     (1, shafter) &  2.317087 &  4.174768 &   0.145166 &  0.185782 \\
45 &  (3, elcajon, 15) &     (2, donovan) &  2.317087 &  4.174768 &   0.145166 &  0.185782 \\
31 &  (3, elcajon, 18) &     (1, shafter) &  2.540051 &  3.222305 &   0.159138 &  0.142937 \\
32 &  (3, elcajon, 18) &     (2, donovan) &  2.540051 &  3.222305 &   0.159138 &  0.142937 \\
19 &  (3, elcajon, 20) &     (2, donovan) &  2.365656 &  3.836465 &   0.148421 &  0.169852 \\
33 &  (3, shafter, 17) &     (2, elcajon) &  3.498404 &  4.454751 &   0.255579 &  0.197187 \\
13 &  (3, shafter, 19) &     (1, donovan) &  3.902972 &  4.688188 &   0.285135 &  0.207520 \\
14 &  (3, shafter, 19) &     (2, elcajon) &  3.902972 &  4.688188 &   0.285135 &  0.207520 \\
34 &  (3, shafter, 21) &     (1, donovan) &  3.050720 &  4.577955 &   0.222873 &  0.202640 \\
35 &  (3, shafter, 21) &     (2, elcajon) &  3.050720 &  4.577955 &   0.222873 &  0.202640 \\
\bottomrule
\end{tabular}

\caption{Level 1 train results for NN[2]}
\end{table}
\begin{table}[H]
\centering
\scriptsize
\begin{tabular}{lllrrrr}
\toprule
{} &             Model & Testing Location &   NO2 MAE &     O3 MAE &  NO2 CvMAE &  O3 CvMAE \\
\midrule
20 &  (1, donovan, 19) &     (2, elcajon) &  5.148282 &  10.367669 &   0.452728 &  0.268582 \\
21 &  (1, donovan, 19) &     (3, shafter) &  8.255221 &  20.391712 &   0.603092 &  0.902626 \\
24 &  (1, donovan, 21) &     (2, elcajon) &  5.777976 &   9.128922 &   0.508102 &  0.236492 \\
25 &  (1, donovan, 21) &     (3, shafter) &  6.595977 &  11.449343 &   0.481875 &  0.506798 \\
18 &  (1, elcajon, 11) &     (2, shafter) &  4.750567 &   8.092037 &   0.350447 &  0.256888 \\
19 &  (1, elcajon, 11) &     (3, donovan) &  6.227742 &   8.673588 &   0.517650 &  0.291224 \\
33 &  (1, elcajon, 12) &     (2, shafter) &  5.338213 &   9.234742 &   0.397692 &  0.293133 \\
34 &  (1, elcajon, 12) &     (3, donovan) &  7.018052 &   9.928023 &   0.583957 &  0.332869 \\
8  &  (1, elcajon, 13) &     (2, shafter) &  5.581877 &  13.827278 &   0.413548 &  0.436338 \\
9  &  (1, elcajon, 13) &     (3, donovan) &  9.793355 &  36.820648 &   0.822203 &  1.237559 \\
4  &  (1, shafter, 15) &     (2, donovan) &  6.539024 &  14.793696 &   0.614035 &  0.344109 \\
5  &  (1, shafter, 15) &     (3, elcajon) &  6.365517 &   9.160675 &   0.398801 &  0.407661 \\
10 &  (1, shafter, 18) &     (2, donovan) &  6.732898 &  19.346482 &   0.638883 &  0.456490 \\
11 &  (1, shafter, 18) &     (3, elcajon) &  6.221446 &  11.651436 &   0.389782 &  0.516842 \\
16 &  (2, donovan, 15) &     (1, shafter) &  6.524377 &  15.971837 &   0.964656 &  0.490681 \\
17 &  (2, donovan, 15) &     (3, elcajon) &  4.936247 &  11.100471 &   0.309257 &  0.493984 \\
22 &  (2, donovan, 18) &     (1, shafter) &  5.418111 &  10.557459 &   0.981920 &  0.322730 \\
23 &  (2, donovan, 18) &     (3, elcajon) &  5.015599 &   9.599606 &   0.314234 &  0.425825 \\
7  &  (2, donovan, 20) &     (3, elcajon) &  4.322025 &   9.056507 &   0.271163 &  0.400960 \\
6  &  (2, elcajon, 17) &     (3, shafter) &  4.459051 &   8.865127 &   0.325760 &  0.392409 \\
12 &  (2, elcajon, 19) &     (1, donovan) &  2.422743 &   5.612289 &   0.572202 &  0.147147 \\
13 &  (2, elcajon, 19) &     (3, shafter) &  5.851506 &  15.364891 &   0.427487 &  0.680117 \\
26 &  (2, elcajon, 21) &     (1, donovan) &  2.549793 &   4.713782 &   0.602209 &  0.123589 \\
27 &  (2, elcajon, 21) &     (3, shafter) &  4.335313 &   8.996973 &   0.316720 &  0.398245 \\
0  &  (3, donovan, 11) &     (1, elcajon) &  4.137962 &   6.929712 &   0.762205 &  0.183276 \\
1  &  (3, donovan, 11) &     (2, shafter) &  4.871891 &   8.493657 &   0.359397 &  0.269638 \\
2  &  (3, donovan, 12) &     (1, elcajon) &  3.350593 &  15.302875 &   0.617173 &  0.404729 \\
3  &  (3, donovan, 12) &     (2, shafter) &  5.446244 &  10.061753 &   0.405740 &  0.319385 \\
31 &  (3, donovan, 13) &     (1, elcajon) &  3.623134 &   7.152712 &   0.667375 &  0.189174 \\
32 &  (3, donovan, 13) &     (2, shafter) &  4.964074 &  10.002417 &   0.367776 &  0.315639 \\
14 &  (3, elcajon, 15) &     (1, shafter) &  4.572721 &   9.766020 &   0.676095 &  0.300028 \\
15 &  (3, elcajon, 15) &     (2, donovan) &  5.636174 &   8.506492 &   0.529255 &  0.197866 \\
29 &  (3, elcajon, 18) &     (1, shafter) &  3.488751 &   6.089733 &   0.632264 &  0.186156 \\
30 &  (3, elcajon, 18) &     (2, donovan) &  5.892300 &   8.379243 &   0.559118 &  0.197712 \\
28 &  (3, elcajon, 20) &     (2, donovan) &  5.949646 &   8.648831 &   0.564560 &  0.204073 \\
\bottomrule
\end{tabular}

\caption{Level 1 test results for NN[2]}
\end{table}

\begin{table}[H]
\centering
\scriptsize
\begin{tabular}{llrrrr}
\toprule
{} &                               Model &   NO2 MAE &    O3 MAE &  NO2 CvMAE &  O3 CvMAE \\
\midrule
0  &  (15, \{(1, shafter), (3, elcajon)\}) &  2.572285 &  5.376839 &   0.188166 &  0.215228 \\
1  &  (18, \{(1, shafter), (3, elcajon)\}) &  2.590769 &  3.808646 &   0.172008 &  0.162628 \\
2  &  (15, \{(2, donovan), (3, elcajon)\}) &  3.638093 &  5.103324 &   0.250176 &  0.182559 \\
3  &  (15, \{(2, donovan), (3, elcajon)\}) &  3.498899 &  5.058574 &   0.240604 &  0.180958 \\
4  &  (15, \{(2, donovan), (1, shafter)\}) &  4.160162 &  6.446772 &   0.472848 &  0.169575 \\
5  &  (18, \{(1, shafter), (3, elcajon)\}) &  2.689220 &  3.791213 &   0.178544 &  0.161884 \\
6  &  (15, \{(2, donovan), (1, shafter)\}) &  4.650485 &  7.008153 &   0.528579 &  0.184341 \\
7  &  (18, \{(2, donovan), (3, elcajon)\}) &  3.518105 &  4.608240 &   0.242692 &  0.165150 \\
8  &  (15, \{(1, shafter), (3, elcajon)\}) &  2.454029 &  4.381762 &   0.179516 &  0.175396 \\
9  &  (18, \{(2, donovan), (1, shafter)\}) &  6.248669 &  6.467920 &   0.656383 &  0.160020 \\
10 &  (18, \{(2, donovan), (1, shafter)\}) &  5.674383 &  5.988172 &   0.596058 &  0.148151 \\
11 &  (18, \{(2, donovan), (3, elcajon)\}) &  3.566501 &  5.359876 &   0.246030 &  0.192087 \\
\bottomrule
\end{tabular}

\caption{Level 2 train results for NN[2]}
\end{table}
\begin{table}[H]
\centering
\scriptsize
\begin{tabular}{lrrrrr}
\toprule
{} &  Model &   NO2 MAE &     O3 MAE &  NO2 CvMAE &  O3 CvMAE \\
\midrule
0 &   18.0 &  6.154747 &   7.835935 &   0.590986 &  0.183991 \\
1 &   15.0 &  6.382545 &  14.575332 &   0.769911 &  0.442560 \\
2 &   15.0 &  3.913819 &   9.565233 &   0.246413 &  0.424429 \\
3 &   18.0 &  5.968737 &  10.898183 &   0.750799 &  0.323601 \\
4 &   15.0 &  5.736707 &  10.530624 &   0.539798 &  0.245204 \\
5 &   18.0 &  4.827971 &  10.490020 &   0.302681 &  0.465884 \\
\bottomrule
\end{tabular}

\caption{Level 2 test results for NN[2]}
\end{table}

\begin{table}[H]
\centering
\scriptsize
\begin{tabular}{lrrrrr}
\toprule
{} &  Model &   NO2 MAE &    O3 MAE &  NO2 CvMAE &  O3 CvMAE \\
\midrule
0 &   20.0 &  3.295158 &  5.038287 &   0.228062 &  0.180321 \\
1 &   21.0 &  2.571175 &  5.067573 &   0.298225 &  0.133160 \\
2 &   18.0 &  3.583515 &  4.792348 &   0.258286 &  0.169539 \\
3 &   19.0 &  2.404447 &  5.428438 &   0.278887 &  0.142642 \\
4 &   15.0 &  3.260104 &  5.460390 &   0.251641 &  0.188680 \\
5 &   17.0 &  2.803051 &  5.373645 &   0.248987 &  0.140890 \\
6 &   11.0 &  4.283376 &  4.895855 &   0.404398 &  0.155636 \\
7 &   12.0 &  2.986514 &  5.105302 &   0.281891 &  0.162341 \\
8 &   13.0 &  3.276111 &  5.092421 &   0.315272 &  0.161854 \\
\bottomrule
\end{tabular}

\caption{Level 3 train results for NN[2]}
\end{table}
\begin{table}[H]
\centering
\scriptsize
\begin{tabular}{lrrrrr}
\toprule
{} &  Model &   NO2 MAE &    O3 MAE &  NO2 CvMAE &  O3 CvMAE \\
\midrule
0 &   20.0 &  3.318197 &  5.019965 &   0.229161 &  0.179701 \\
1 &   21.0 &  2.637760 &  5.174456 &   0.302504 &  0.134645 \\
2 &   18.0 &  3.635188 &  4.776389 &   0.261178 &  0.169298 \\
3 &   19.0 &  2.466077 &  5.481129 &   0.282815 &  0.142625 \\
4 &   15.0 &  3.233792 &  5.458021 &   0.248354 &  0.189165 \\
5 &   17.0 &  2.821470 &  5.370093 &   0.250848 &  0.138757 \\
6 &   11.0 &  4.363108 &  4.948445 &   0.406110 &  0.157854 \\
7 &   12.0 &  3.074722 &  5.117690 &   0.286462 &  0.163075 \\
8 &   13.0 &  3.373182 &  5.202649 &   0.320035 &  0.165390 \\
\bottomrule
\end{tabular}

\caption{Level 3 test results for NN[2]}
\end{table}

\subsection{Benchmarks for NN[4]}
\label{sec:results-nn4}

\begin{table}[H]
\centering
\scriptsize
\begin{tabular}{lllrrrr}
\toprule
{} &             Model &  Testing Location &   NO2 MAE &    O3 MAE &  NO2 CvMAE &  O3 CvMAE \\
\midrule
16 &  (1, donovan, 19) &  (1, donovan, 19) &  1.598798 &  2.655370 &   0.379873 &  0.070111 \\
5  &  (1, donovan, 21) &  (1, donovan, 21) &  1.579410 &  2.631647 &   0.375266 &  0.069484 \\
8  &  (1, elcajon, 11) &  (1, elcajon, 11) &  1.090424 &  3.075850 &   0.203362 &  0.081263 \\
2  &  (1, elcajon, 12) &  (1, elcajon, 12) &  1.073017 &  2.392930 &   0.200116 &  0.063221 \\
9  &  (1, elcajon, 13) &  (1, elcajon, 13) &  1.149510 &  2.716713 &   0.214382 &  0.071775 \\
0  &  (1, shafter, 15) &  (1, shafter, 15) &  1.839555 &  3.078209 &   0.272408 &  0.093814 \\
6  &  (1, shafter, 18) &  (1, shafter, 18) &  1.133296 &  2.087040 &   0.202950 &  0.063455 \\
14 &  (2, donovan, 15) &  (2, donovan, 15) &  3.620364 &  4.510126 &   0.340834 &  0.105045 \\
22 &  (2, donovan, 18) &  (2, donovan, 18) &  3.867774 &  4.618096 &   0.372499 &  0.108303 \\
4  &  (2, donovan, 20) &  (2, donovan, 20) &  3.535354 &  4.564998 &   0.340484 &  0.107057 \\
13 &  (2, elcajon, 17) &  (2, elcajon, 17) &  2.060804 &  3.941635 &   0.183055 &  0.103344 \\
1  &  (2, elcajon, 19) &  (2, elcajon, 19) &  2.090866 &  4.521853 &   0.186182 &  0.118484 \\
3  &  (2, elcajon, 21) &  (2, elcajon, 21) &  2.049768 &  3.927825 &   0.182522 &  0.102919 \\
20 &  (2, shafter, 11) &  (2, shafter, 11) &  3.215755 &  4.713431 &   0.239533 &  0.149387 \\
21 &  (2, shafter, 12) &  (2, shafter, 12) &  3.465279 &  4.671965 &   0.257798 &  0.147744 \\
11 &  (2, shafter, 13) &  (2, shafter, 13) &  3.077443 &  4.555251 &   0.226935 &  0.145039 \\
23 &  (3, donovan, 11) &  (3, donovan, 11) &  3.134097 &  3.745577 &   0.264293 &  0.125233 \\
15 &  (3, donovan, 12) &  (3, donovan, 12) &  2.598337 &  3.872787 &   0.219052 &  0.129535 \\
12 &  (3, donovan, 13) &  (3, donovan, 13) &  2.858027 &  3.910363 &   0.243435 &  0.131442 \\
24 &  (3, elcajon, 15) &  (3, elcajon, 15) &  2.195624 &  3.413974 &   0.138407 &  0.151375 \\
17 &  (3, elcajon, 18) &  (3, elcajon, 18) &  2.072834 &  2.718973 &   0.129975 &  0.120792 \\
10 &  (3, elcajon, 20) &  (3, elcajon, 20) &  1.909335 &  2.984222 &   0.119680 &  0.132642 \\
18 &  (3, shafter, 17) &  (3, shafter, 17) &  2.910655 &  3.553848 &   0.212643 &  0.155850 \\
7  &  (3, shafter, 19) &  (3, shafter, 19) &  3.250766 &  4.017558 &   0.237490 &  0.176186 \\
19 &  (3, shafter, 21) &  (3, shafter, 21) &  2.507448 &  3.801003 &   0.183186 &  0.166689 \\
\bottomrule
\end{tabular}

\caption{Level 0 train results for NN[4]}
\end{table}
\begin{table}[H]
\centering
\scriptsize
\begin{tabular}{lllrrrr}
\toprule
{} &             Model &  Testing Location &   NO2 MAE &    O3 MAE &  NO2 CvMAE &  O3 CvMAE \\
\midrule
16 &  (1, donovan, 19) &  (1, donovan, 19) &  1.646639 &  2.677330 &   0.388902 &  0.070196 \\
5  &  (1, donovan, 21) &  (1, donovan, 21) &  1.639563 &  2.700179 &   0.387231 &  0.070795 \\
8  &  (1, elcajon, 11) &  (1, elcajon, 11) &  1.131059 &  3.251013 &   0.208339 &  0.085982 \\
2  &  (1, elcajon, 12) &  (1, elcajon, 12) &  1.107206 &  2.463823 &   0.203945 &  0.065163 \\
9  &  (1, elcajon, 13) &  (1, elcajon, 13) &  1.204358 &  2.809111 &   0.221840 &  0.074295 \\
0  &  (1, shafter, 15) &  (1, shafter, 15) &  1.856185 &  3.185708 &   0.274445 &  0.097870 \\
6  &  (1, shafter, 18) &  (1, shafter, 18) &  1.256305 &  2.275298 &   0.227679 &  0.069553 \\
14 &  (2, donovan, 15) &  (2, donovan, 15) &  3.680800 &  4.625006 &   0.345639 &  0.107580 \\
22 &  (2, donovan, 18) &  (2, donovan, 18) &  4.032552 &  4.812236 &   0.382648 &  0.113547 \\
4  &  (2, donovan, 20) &  (2, donovan, 20) &  3.619083 &  4.609697 &   0.343414 &  0.108768 \\
13 &  (2, elcajon, 17) &  (2, elcajon, 17) &  2.130942 &  4.010484 &   0.189456 &  0.103626 \\
1  &  (2, elcajon, 19) &  (2, elcajon, 19) &  2.164384 &  4.600333 &   0.190331 &  0.119175 \\
3  &  (2, elcajon, 21) &  (2, elcajon, 21) &  2.171034 &  4.047821 &   0.190916 &  0.104862 \\
20 &  (2, shafter, 11) &  (2, shafter, 11) &  3.204198 &  4.767020 &   0.236372 &  0.151333 \\
21 &  (2, shafter, 12) &  (2, shafter, 12) &  3.546977 &  4.753282 &   0.264247 &  0.150881 \\
11 &  (2, shafter, 13) &  (2, shafter, 13) &  3.124168 &  4.599609 &   0.231462 &  0.145147 \\
23 &  (3, donovan, 11) &  (3, donovan, 11) &  3.227730 &  3.858718 &   0.268289 &  0.129560 \\
15 &  (3, donovan, 12) &  (3, donovan, 12) &  2.697892 &  3.924738 &   0.224486 &  0.131589 \\
12 &  (3, donovan, 13) &  (3, donovan, 13) &  2.972283 &  4.047961 &   0.249538 &  0.136054 \\
24 &  (3, elcajon, 15) &  (3, elcajon, 15) &  2.227096 &  3.465248 &   0.139528 &  0.154208 \\
17 &  (3, elcajon, 18) &  (3, elcajon, 18) &  2.116246 &  2.784545 &   0.132586 &  0.123519 \\
10 &  (3, elcajon, 20) &  (3, elcajon, 20) &  1.943790 &  3.016136 &   0.121953 &  0.133534 \\
18 &  (3, shafter, 17) &  (3, shafter, 17) &  2.934812 &  3.598334 &   0.214405 &  0.159278 \\
7  &  (3, shafter, 19) &  (3, shafter, 19) &  3.235436 &  3.980017 &   0.236368 &  0.176173 \\
19 &  (3, shafter, 21) &  (3, shafter, 21) &  2.511674 &  3.819862 &   0.183492 &  0.169084 \\
\bottomrule
\end{tabular}

\caption{Level 0 test results for NN[4]}
\end{table}

\begin{table}[H]
\centering
\scriptsize
\begin{tabular}{llrrrr}
\toprule
{} &             Model &   NO2 MAE &    O3 MAE &  NO2 CvMAE &  O3 CvMAE \\
\midrule
0  &  (3, donovan, 11) &  3.030522 &  3.838621 &   0.251897 &  0.128885 \\
1  &  (3, donovan, 12) &  2.761807 &  4.234938 &   0.229804 &  0.141990 \\
2  &  (1, shafter, 15) &  1.803251 &  3.118441 &   0.266618 &  0.095804 \\
3  &  (2, donovan, 20) &  3.710424 &  4.666968 &   0.352081 &  0.110119 \\
4  &  (1, elcajon, 13) &  1.192207 &  2.642059 &   0.219602 &  0.069877 \\
5  &  (1, shafter, 18) &  1.174105 &  2.001298 &   0.212782 &  0.061177 \\
6  &  (2, elcajon, 19) &  2.276630 &  4.595535 &   0.200202 &  0.119051 \\
7  &  (3, elcajon, 15) &  1.949285 &  3.029513 &   0.122123 &  0.134817 \\
8  &  (2, donovan, 15) &  4.018568 &  5.240196 &   0.377356 &  0.121890 \\
9  &  (1, elcajon, 11) &  1.136106 &  3.274470 &   0.209269 &  0.086603 \\
10 &  (1, donovan, 19) &  1.571407 &  2.618338 &   0.371134 &  0.068649 \\
11 &  (2, donovan, 18) &  4.015557 &  4.562081 &   0.381035 &  0.107645 \\
12 &  (1, donovan, 21) &  1.448614 &  2.553764 &   0.342133 &  0.066956 \\
13 &  (2, elcajon, 21) &  2.191676 &  4.118526 &   0.192731 &  0.106694 \\
14 &  (3, elcajon, 20) &  2.011536 &  3.024782 &   0.126204 &  0.133917 \\
15 &  (3, elcajon, 18) &  2.168078 &  2.652108 &   0.135833 &  0.117644 \\
16 &  (3, donovan, 13) &  3.089183 &  4.123722 &   0.259353 &  0.138600 \\
17 &  (1, elcajon, 12) &  1.186934 &  3.125216 &   0.218631 &  0.082655 \\
\bottomrule
\end{tabular}

\caption{Level 1 train results for NN[4]}
\end{table}
\begin{table}[H]
\centering
\scriptsize
\begin{tabular}{lllrrrr}
\toprule
{} &             Model & Testing Location &    NO2 MAE &     O3 MAE &  NO2 CvMAE &  O3 CvMAE \\
\midrule
29 &  (1, donovan, 19) &     (2, elcajon) &   6.536488 &  12.553209 &   0.574804 &  0.325200 \\
30 &  (1, donovan, 19) &     (3, shafter) &  13.700890 &  27.299761 &   1.000930 &  1.208406 \\
9  &  (1, donovan, 21) &     (2, elcajon) &   5.214600 &  10.052968 &   0.458560 &  0.260430 \\
10 &  (1, donovan, 21) &     (3, shafter) &  12.256566 &  17.817916 &   0.895414 &  0.788698 \\
15 &  (1, elcajon, 11) &     (2, shafter) &   5.422440 &  10.004261 &   0.400011 &  0.317593 \\
16 &  (1, elcajon, 11) &     (3, donovan) &   8.208423 &   8.845048 &   0.682284 &  0.296981 \\
4  &  (1, elcajon, 12) &     (2, shafter) &   5.608348 &   9.634313 &   0.417817 &  0.305817 \\
5  &  (1, elcajon, 12) &     (3, donovan) &   8.477711 &  10.724476 &   0.705412 &  0.359572 \\
17 &  (1, elcajon, 13) &     (2, shafter) &   6.514591 &  12.232702 &   0.482650 &  0.386019 \\
18 &  (1, elcajon, 13) &     (3, donovan) &   8.772513 &  25.361335 &   0.736498 &  0.852406 \\
0  &  (1, shafter, 15) &     (2, donovan) &   7.423044 &  17.263673 &   0.697048 &  0.401562 \\
1  &  (1, shafter, 15) &     (3, elcajon) &   8.519809 &  10.013927 &   0.533767 &  0.445632 \\
11 &  (1, shafter, 18) &     (2, donovan) &   8.798300 &  14.402095 &   0.834868 &  0.339824 \\
12 &  (1, shafter, 18) &     (3, elcajon) &   9.134043 &   8.499010 &   0.572260 &  0.377004 \\
25 &  (2, donovan, 15) &     (1, shafter) &   9.677780 &  16.192526 &   1.430899 &  0.497461 \\
26 &  (2, donovan, 15) &     (3, elcajon) &  13.446282 &  19.687384 &   0.842412 &  0.876112 \\
40 &  (2, donovan, 18) &     (1, shafter) &   8.160203 &  11.658369 &   1.478867 &  0.356384 \\
41 &  (2, donovan, 18) &     (3, elcajon) &   9.481830 &  14.012520 &   0.594049 &  0.621576 \\
8  &  (2, donovan, 20) &     (3, elcajon) &  10.685807 &  17.723073 &   0.670426 &  0.784656 \\
24 &  (2, elcajon, 17) &     (3, shafter) &   5.529337 &   8.433496 &   0.403950 &  0.373303 \\
2  &  (2, elcajon, 19) &     (1, donovan) &   2.475568 &   5.260309 &   0.584678 &  0.137919 \\
3  &  (2, elcajon, 19) &     (3, shafter) &   6.018672 &  10.428402 &   0.439699 &  0.461606 \\
6  &  (2, elcajon, 21) &     (1, donovan) &   2.689363 &   5.657763 &   0.635172 &  0.148339 \\
7  &  (2, elcajon, 21) &     (3, shafter) &   4.932580 &   8.294777 &   0.360354 &  0.367163 \\
36 &  (2, shafter, 11) &     (1, elcajon) &   4.162390 &  11.913171 &   0.766705 &  0.315078 \\
37 &  (2, shafter, 11) &     (3, donovan) &   6.892461 &  13.430554 &   0.572901 &  0.450944 \\
38 &  (2, shafter, 12) &     (1, elcajon) &   7.077577 &   7.467346 &   1.303677 &  0.197496 \\
39 &  (2, shafter, 12) &     (3, donovan) &   7.651623 &  10.861686 &   0.636675 &  0.364173 \\
20 &  (2, shafter, 13) &     (1, elcajon) &   4.690778 &  16.533990 &   0.864033 &  0.437289 \\
21 &  (2, shafter, 13) &     (3, donovan) &   9.141173 &  16.189858 &   0.767448 &  0.544149 \\
42 &  (3, donovan, 11) &     (1, elcajon) &   4.724865 &   6.984454 &   0.870312 &  0.184724 \\
43 &  (3, donovan, 11) &     (2, shafter) &   5.837603 &   9.234832 &   0.430637 &  0.293167 \\
27 &  (3, donovan, 12) &     (1, elcajon) &   5.174095 &  11.562465 &   0.953059 &  0.305803 \\
28 &  (3, donovan, 12) &     (2, shafter) &   5.907204 &  10.736715 &   0.440081 &  0.340810 \\
22 &  (3, donovan, 13) &     (1, elcajon) &   5.288720 &   8.103993 &   0.974173 &  0.214334 \\
23 &  (3, donovan, 13) &     (2, shafter) &   6.176490 &  10.697359 &   0.457601 &  0.337569 \\
44 &  (3, elcajon, 15) &     (1, shafter) &   3.710342 &   9.753052 &   0.548589 &  0.299630 \\
45 &  (3, elcajon, 15) &     (2, donovan) &   5.661848 &   9.526385 &   0.531666 &  0.221589 \\
31 &  (3, elcajon, 18) &     (1, shafter) &   3.642625 &   7.919467 &   0.660150 &  0.242089 \\
32 &  (3, elcajon, 18) &     (2, donovan) &   5.831023 &   8.934960 &   0.553304 &  0.210825 \\
19 &  (3, elcajon, 20) &     (2, donovan) &   5.605986 &   9.792729 &   0.531950 &  0.231064 \\
33 &  (3, shafter, 17) &     (2, elcajon) &   7.525477 &  10.878168 &   0.669067 &  0.281080 \\
13 &  (3, shafter, 19) &     (1, donovan) &   2.731428 &   6.203572 &   0.645107 &  0.162650 \\
14 &  (3, shafter, 19) &     (2, elcajon) &   5.168947 &   8.853443 &   0.454545 &  0.229355 \\
34 &  (3, shafter, 21) &     (1, donovan) &   3.538180 &   5.242839 &   0.835645 &  0.137460 \\
35 &  (3, shafter, 21) &     (2, elcajon) &   4.334550 &   8.161131 &   0.381170 &  0.211420 \\
\bottomrule
\end{tabular}

\caption{Level 1 test results for NN[4]}
\end{table}

\begin{table}[H]
\centering
\scriptsize
\begin{tabular}{llrrrr}
\toprule
{} &                               Model &   NO2 MAE &    O3 MAE &  NO2 CvMAE &  O3 CvMAE \\
\midrule
0  &  (13, \{(3, donovan), (2, shafter)\}) &  3.370171 &  4.823963 &   0.271147 &  0.158760 \\
1  &  (12, \{(3, donovan), (1, elcajon)\}) &  2.950991 &  5.682814 &   0.274935 &  0.181083 \\
2  &  (19, \{(1, donovan), (3, shafter)\}) &  3.115379 &  4.201793 &   0.259471 &  0.165705 \\
3  &  (15, \{(3, elcajon), (2, donovan)\}) &  3.008622 &  4.273620 &   0.206890 &  0.152879 \\
4  &  (19, \{(2, elcajon), (3, shafter)\}) &  3.478385 &  4.752491 &   0.266184 &  0.176800 \\
5  &  (19, \{(2, elcajon), (1, donovan)\}) &  2.066151 &  4.039185 &   0.236950 &  0.105104 \\
6  &  (13, \{(2, shafter), (1, elcajon)\}) &  2.668262 &  4.658689 &   0.251145 &  0.137550 \\
7  &  (11, \{(2, shafter), (1, elcajon)\}) &  2.664860 &  4.835774 &   0.250129 &  0.143268 \\
8  &  (11, \{(3, donovan), (2, shafter)\}) &  3.360152 &  4.849313 &   0.268944 &  0.160019 \\
9  &  (12, \{(3, donovan), (2, shafter)\}) &  3.179092 &  5.337817 &   0.255191 &  0.175871 \\
10 &  (21, \{(2, elcajon), (1, donovan)\}) &  2.117772 &  4.094078 &   0.242870 &  0.106533 \\
11 &  (15, \{(1, shafter), (2, donovan)\}) &  3.294208 &  5.478238 &   0.374423 &  0.144099 \\
12 &  (15, \{(1, shafter), (3, elcajon)\}) &  2.199234 &  3.437771 &   0.160877 &  0.137609 \\
13 &  (11, \{(3, donovan), (1, elcajon)\}) &  2.698117 &  4.004875 &   0.251136 &  0.127755 \\
14 &  (12, \{(2, shafter), (1, elcajon)\}) &  2.741305 &  4.409537 &   0.257455 &  0.130874 \\
15 &  (13, \{(3, donovan), (1, elcajon)\}) &  2.850300 &  4.108897 &   0.270426 &  0.130620 \\
16 &  (21, \{(2, elcajon), (3, shafter)\}) &  2.792276 &  4.338721 &   0.213680 &  0.161407 \\
17 &  (18, \{(1, shafter), (2, donovan)\}) &  4.058987 &  4.831247 &   0.426371 &  0.119528 \\
18 &  (21, \{(1, donovan), (3, shafter)\}) &  2.558208 &  3.843982 &   0.213066 &  0.151594 \\
19 &  (18, \{(1, shafter), (3, elcajon)\}) &  2.213259 &  2.843124 &   0.146944 &  0.121401 \\
20 &  (18, \{(3, elcajon), (2, donovan)\}) &  2.959820 &  3.710114 &   0.204179 &  0.132963 \\
\bottomrule
\end{tabular}

\caption{Level 2 train results for NN[4]}
\end{table}
\begin{table}[H]
\centering
\scriptsize
\begin{tabular}{lrrrrr}
\toprule
{} &  Model &   NO2 MAE &     O3 MAE &  NO2 CvMAE &  O3 CvMAE \\
\midrule
0 &   15.0 &  5.654460 &  11.868509 &   0.532059 &  0.276356 \\
1 &   15.0 &  5.929237 &  14.048132 &   0.715230 &  0.426552 \\
2 &   18.0 &  5.717043 &   8.398271 &   0.548957 &  0.197195 \\
3 &   15.0 &  9.565594 &  11.242872 &   0.602248 &  0.498869 \\
4 &   18.0 &  7.466347 &  13.357641 &   0.468090 &  0.593241 \\
5 &   18.0 &  7.100896 &  10.778919 &   0.893211 &  0.320060 \\
\bottomrule
\end{tabular}

\caption{Level 2 test results for NN[4]}
\end{table}

\begin{table}[H]
\centering
\scriptsize
\begin{tabular}{lllrrrr}
\toprule
{} & Model &          Test &   NO2 MAE &    O3 MAE &  NO2 CvMAE &  O3 CvMAE \\
\midrule
0  &    11 &  (2, shafter) &  3.803538 &  5.549903 &   0.283315 &  0.175898 \\
1  &    11 &  (3, donovan) &  3.054310 &  4.151130 &   0.257564 &  0.138792 \\
2  &    11 &  (1, elcajon) &  2.296139 &  4.627076 &   0.428226 &  0.122247 \\
3  &    17 &  (2, elcajon) &  2.697029 &  4.757676 &   0.239570 &  0.124740 \\
4  &    17 &  (3, shafter) &  3.179543 &  3.712308 &   0.232287 &  0.162799 \\
5  &    18 &  (1, shafter) &  3.502196 &  4.946127 &   0.627170 &  0.150384 \\
6  &    18 &  (3, elcajon) &  2.602726 &  2.991688 &   0.163201 &  0.132907 \\
7  &    18 &  (2, donovan) &  4.943300 &  5.840910 &   0.476081 &  0.136980 \\
8  &    13 &  (2, shafter) &  3.698847 &  5.887221 &   0.272758 &  0.187448 \\
9  &    13 &  (3, donovan) &  3.262873 &  4.653230 &   0.277918 &  0.156412 \\
10 &    13 &  (1, elcajon) &  1.930914 &  4.264095 &   0.360112 &  0.112657 \\
11 &    12 &  (2, shafter) &  3.624222 &  5.723144 &   0.269622 &  0.180986 \\
12 &    12 &  (3, donovan) &  2.921068 &  4.779876 &   0.246260 &  0.159875 \\
13 &    12 &  (1, elcajon) &  1.900706 &  4.465228 &   0.354478 &  0.117971 \\
14 &    20 &  (3, elcajon) &  2.423089 &  3.718384 &   0.151883 &  0.165274 \\
15 &    20 &  (2, donovan) &  4.802678 &  6.190695 &   0.462538 &  0.145183 \\
16 &    15 &  (1, shafter) &  2.656244 &  4.699894 &   0.393347 &  0.143238 \\
17 &    15 &  (3, elcajon) &  2.511351 &  3.678134 &   0.158310 &  0.163088 \\
18 &    15 &  (2, donovan) &  5.136696 &  6.775231 &   0.483587 &  0.157801 \\
19 &    19 &  (2, elcajon) &  2.861018 &  5.902118 &   0.254760 &  0.154651 \\
20 &    19 &  (1, donovan) &  2.409992 &  3.858104 &   0.572612 &  0.101867 \\
21 &    19 &  (3, shafter) &  3.334381 &  4.154406 &   0.243599 &  0.182187 \\
22 &    21 &  (2, elcajon) &  2.577120 &  4.999255 &   0.229481 &  0.130994 \\
23 &    21 &  (1, donovan) &  2.025689 &  3.832054 &   0.481302 &  0.101179 \\
24 &    21 &  (3, shafter) &  2.684415 &  4.075558 &   0.196115 &  0.178729 \\
\bottomrule
\end{tabular}

\caption{Level 3 train results for NN[4]}
\end{table}
\begin{table}[H]
\centering
\scriptsize
\begin{tabular}{lrrrrr}
\toprule
{} &  Model &   NO2 MAE &    O3 MAE &  NO2 CvMAE &  O3 CvMAE \\
\midrule
0 &   20.0 &  2.973492 &  4.134780 &   0.204743 &  0.147969 \\
1 &   21.0 &  2.075435 &  3.895733 &   0.240403 &  0.102053 \\
2 &   18.0 &  3.193144 &  4.124167 &   0.227443 &  0.146965 \\
3 &   19.0 &  2.080901 &  4.872970 &   0.241036 &  0.127653 \\
4 &   15.0 &  3.380655 &  5.236747 &   0.256225 &  0.180905 \\
5 &   17.0 &  2.051424 &  3.899039 &   0.182386 &  0.100747 \\
6 &   11.0 &  2.685447 &  3.944221 &   0.248520 &  0.126044 \\
7 &   12.0 &  2.520655 &  4.165288 &   0.235248 &  0.132999 \\
8 &   13.0 &  2.826849 &  4.350050 &   0.268082 &  0.138723 \\
\bottomrule
\end{tabular}

\caption{Level 3 test results for NN[4]}
\end{table}

\subsection{Benchmarks for Subu}
\label{sec:results-nn4}

\begin{table}[H]
\centering
\scriptsize
\begin{tabular}{lllrrrr}
\toprule
{} &             Model &  Testing Location &   NO2 MAE &    O3 MAE &  NO2 CvMAE &  O3 CvMAE \\
\midrule
16 &  (1, donovan, 19) &  (1, donovan, 19) &  0.337740 &  0.546776 &   0.080247 &  0.014437 \\
5  &  (1, donovan, 21) &  (1, donovan, 21) &  0.323311 &  0.535830 &   0.076818 &  0.014148 \\
8  &  (1, elcajon, 11) &  (1, elcajon, 11) &  0.301001 &  0.513528 &   0.056136 &  0.013567 \\
2  &  (1, elcajon, 12) &  (1, elcajon, 12) &  0.301750 &  0.515531 &   0.056276 &  0.013620 \\
9  &  (1, elcajon, 13) &  (1, elcajon, 13) &  0.306623 &  0.546167 &   0.057185 &  0.014430 \\
0  &  (1, shafter, 15) &  (1, shafter, 15) &  0.323170 &  0.566483 &   0.047856 &  0.017265 \\
6  &  (1, shafter, 18) &  (1, shafter, 18) &  0.290350 &  0.553727 &   0.051996 &  0.016836 \\
14 &  (2, donovan, 15) &  (2, donovan, 15) &  0.730868 &  0.911470 &   0.068807 &  0.021229 \\
22 &  (2, donovan, 18) &  (2, donovan, 18) &  0.720733 &  0.887428 &   0.069413 &  0.020812 \\
4  &  (2, donovan, 20) &  (2, donovan, 20) &  0.753036 &  0.891266 &   0.072524 &  0.020902 \\
13 &  (2, elcajon, 17) &  (2, elcajon, 17) &  0.392235 &  0.644277 &   0.034841 &  0.016892 \\
1  &  (2, elcajon, 19) &  (2, elcajon, 19) &  0.445999 &  0.739044 &   0.039714 &  0.019365 \\
3  &  (2, elcajon, 21) &  (2, elcajon, 21) &  0.432871 &  0.715783 &   0.038545 &  0.018755 \\
20 &  (2, shafter, 11) &  (2, shafter, 11) &  0.447909 &  0.686391 &   0.033364 &  0.021754 \\
21 &  (2, shafter, 12) &  (2, shafter, 12) &  0.434242 &  0.688316 &   0.032305 &  0.021767 \\
11 &  (2, shafter, 13) &  (2, shafter, 13) &  0.451918 &  0.699876 &   0.033325 &  0.022284 \\
23 &  (3, donovan, 11) &  (3, donovan, 11) &  0.650381 &  0.712277 &   0.054845 &  0.023815 \\
15 &  (3, donovan, 12) &  (3, donovan, 12) &  0.664451 &  0.773984 &   0.056016 &  0.025888 \\
12 &  (3, donovan, 13) &  (3, donovan, 13) &  0.683324 &  0.772968 &   0.058203 &  0.025982 \\
24 &  (3, elcajon, 15) &  (3, elcajon, 15) &  0.509041 &  0.623718 &   0.032089 &  0.027656 \\
17 &  (3, elcajon, 18) &  (3, elcajon, 18) &  0.523014 &  0.644509 &   0.032795 &  0.028633 \\
10 &  (3, elcajon, 20) &  (3, elcajon, 20) &  0.500819 &  0.603558 &   0.031392 &  0.026827 \\
18 &  (3, shafter, 17) &  (3, shafter, 17) &  0.507571 &  0.628832 &   0.037081 &  0.027577 \\
7  &  (3, shafter, 19) &  (3, shafter, 19) &  0.568689 &  0.631523 &   0.041547 &  0.027695 \\
19 &  (3, shafter, 21) &  (3, shafter, 21) &  0.541210 &  0.659544 &   0.039539 &  0.028924 \\
\bottomrule
\end{tabular}

\caption{Level 0 train results for Subu}
\end{table}
\begin{table}[H]
\centering
\scriptsize
\begin{tabular}{lllrrrr}
\toprule
{} &             Model &  Testing Location &   NO2 MAE &    O3 MAE &  NO2 CvMAE &  O3 CvMAE \\
\midrule
16 &  (1, donovan, 19) &  (1, donovan, 19) &  0.686943 &  1.089861 &   0.162242 &  0.028575 \\
5  &  (1, donovan, 21) &  (1, donovan, 21) &  0.684105 &  1.103839 &   0.161572 &  0.028941 \\
8  &  (1, elcajon, 11) &  (1, elcajon, 11) &  0.641704 &  1.055547 &   0.118201 &  0.027917 \\
2  &  (1, elcajon, 12) &  (1, elcajon, 12) &  0.641610 &  1.068369 &   0.118183 &  0.028256 \\
9  &  (1, elcajon, 13) &  (1, elcajon, 13) &  0.631901 &  1.119980 &   0.116395 &  0.029621 \\
0  &  (1, shafter, 15) &  (1, shafter, 15) &  0.679875 &  1.183327 &   0.100522 &  0.036354 \\
6  &  (1, shafter, 18) &  (1, shafter, 18) &  0.594978 &  1.066344 &   0.107827 &  0.032597 \\
14 &  (2, donovan, 15) &  (2, donovan, 15) &  1.486924 &  1.773774 &   0.139627 &  0.041259 \\
22 &  (2, donovan, 18) &  (2, donovan, 18) &  1.419766 &  1.732250 &   0.134721 &  0.040873 \\
4  &  (2, donovan, 20) &  (2, donovan, 20) &  1.486669 &  1.757857 &   0.141070 &  0.041477 \\
13 &  (2, elcajon, 17) &  (2, elcajon, 17) &  0.817639 &  1.318731 &   0.072694 &  0.034075 \\
1  &  (2, elcajon, 19) &  (2, elcajon, 19) &  0.944004 &  1.510398 &   0.083014 &  0.039128 \\
3  &  (2, elcajon, 21) &  (2, elcajon, 21) &  0.915845 &  1.486031 &   0.080537 &  0.038497 \\
20 &  (2, shafter, 11) &  (2, shafter, 11) &  0.874304 &  1.308987 &   0.064497 &  0.041555 \\
21 &  (2, shafter, 12) &  (2, shafter, 12) &  0.894952 &  1.383831 &   0.066673 &  0.043926 \\
11 &  (2, shafter, 13) &  (2, shafter, 13) &  0.902136 &  1.363251 &   0.066837 &  0.043019 \\
23 &  (3, donovan, 11) &  (3, donovan, 11) &  1.351965 &  1.470374 &   0.112375 &  0.049369 \\
15 &  (3, donovan, 12) &  (3, donovan, 12) &  1.393470 &  1.564269 &   0.115948 &  0.052447 \\
12 &  (3, donovan, 13) &  (3, donovan, 13) &  1.458868 &  1.615901 &   0.122480 &  0.054311 \\
24 &  (3, elcajon, 15) &  (3, elcajon, 15) &  1.012644 &  1.253725 &   0.063442 &  0.055792 \\
17 &  (3, elcajon, 18) &  (3, elcajon, 18) &  1.069426 &  1.314976 &   0.067001 &  0.058331 \\
10 &  (3, elcajon, 20) &  (3, elcajon, 20) &  1.002429 &  1.211804 &   0.062892 &  0.053650 \\
18 &  (3, shafter, 17) &  (3, shafter, 17) &  1.005406 &  1.272671 &   0.073451 &  0.056334 \\
7  &  (3, shafter, 19) &  (3, shafter, 19) &  1.122436 &  1.269373 &   0.082001 &  0.056188 \\
19 &  (3, shafter, 21) &  (3, shafter, 21) &  1.070099 &  1.312395 &   0.078177 &  0.058092 \\
\bottomrule
\end{tabular}

\caption{Level 0 test results for Subu}
\end{table}

\begin{table}[H]
\centering
\scriptsize
\begin{tabular}{lllrrrr}
\toprule
{} &             Model & Testing Location &   NO2 MAE &    O3 MAE &  NO2 CvMAE &  O3 CvMAE \\
\midrule
20 &  (1, donovan, 19) &     (2, elcajon) &  0.687997 &  1.091423 &   0.162491 &  0.028616 \\
21 &  (1, donovan, 19) &     (3, shafter) &  0.687997 &  1.091423 &   0.162491 &  0.028616 \\
24 &  (1, donovan, 21) &     (2, elcajon) &  0.684494 &  1.103336 &   0.161663 &  0.028928 \\
25 &  (1, donovan, 21) &     (3, shafter) &  0.684494 &  1.103336 &   0.161663 &  0.028928 \\
18 &  (1, elcajon, 11) &     (2, shafter) &  0.640256 &  1.057737 &   0.117934 &  0.027975 \\
19 &  (1, elcajon, 11) &     (3, donovan) &  0.640256 &  1.057737 &   0.117934 &  0.027975 \\
33 &  (1, elcajon, 12) &     (2, shafter) &  0.644690 &  1.063028 &   0.118751 &  0.028115 \\
34 &  (1, elcajon, 12) &     (3, donovan) &  0.644690 &  1.063028 &   0.118751 &  0.028115 \\
8  &  (1, elcajon, 13) &     (2, shafter) &  0.630628 &  1.114752 &   0.116160 &  0.029483 \\
9  &  (1, elcajon, 13) &     (3, donovan) &  0.630628 &  1.114752 &   0.116160 &  0.029483 \\
4  &  (1, shafter, 15) &     (2, donovan) &  0.682778 &  1.178485 &   0.100951 &  0.036205 \\
5  &  (1, shafter, 15) &     (3, elcajon) &  0.682778 &  1.178485 &   0.100951 &  0.036205 \\
10 &  (1, shafter, 18) &     (2, donovan) &  0.595840 &  1.065488 &   0.107984 &  0.032571 \\
11 &  (1, shafter, 18) &     (3, elcajon) &  0.595840 &  1.065488 &   0.107984 &  0.032571 \\
16 &  (2, donovan, 15) &     (1, shafter) &  1.484875 &  1.777398 &   0.139435 &  0.041343 \\
17 &  (2, donovan, 15) &     (3, elcajon) &  1.484875 &  1.777398 &   0.139435 &  0.041343 \\
22 &  (2, donovan, 18) &     (1, shafter) &  1.415028 &  1.734306 &   0.134272 &  0.040922 \\
23 &  (2, donovan, 18) &     (3, elcajon) &  1.415028 &  1.734306 &   0.134272 &  0.040922 \\
7  &  (2, donovan, 20) &     (3, elcajon) &  1.476785 &  1.757999 &   0.140132 &  0.041481 \\
6  &  (2, elcajon, 17) &     (3, shafter) &  0.815422 &  1.319097 &   0.072497 &  0.034084 \\
12 &  (2, elcajon, 19) &     (1, donovan) &  0.946357 &  1.505360 &   0.083220 &  0.038998 \\
13 &  (2, elcajon, 19) &     (3, shafter) &  0.946357 &  1.505360 &   0.083220 &  0.038998 \\
26 &  (2, elcajon, 21) &     (1, donovan) &  0.916909 &  1.476047 &   0.080631 &  0.038238 \\
27 &  (2, elcajon, 21) &     (3, shafter) &  0.916909 &  1.476047 &   0.080631 &  0.038238 \\
0  &  (3, donovan, 11) &     (1, elcajon) &  1.349956 &  1.468746 &   0.112208 &  0.049315 \\
1  &  (3, donovan, 11) &     (2, shafter) &  1.349956 &  1.468746 &   0.112208 &  0.049315 \\
2  &  (3, donovan, 12) &     (1, elcajon) &  1.388455 &  1.564851 &   0.115530 &  0.052467 \\
3  &  (3, donovan, 12) &     (2, shafter) &  1.388455 &  1.564851 &   0.115530 &  0.052467 \\
31 &  (3, donovan, 13) &     (1, elcajon) &  1.458304 &  1.620677 &   0.122432 &  0.054472 \\
32 &  (3, donovan, 13) &     (2, shafter) &  1.458304 &  1.620677 &   0.122432 &  0.054472 \\
14 &  (3, elcajon, 15) &     (1, shafter) &  1.012843 &  1.253126 &   0.063455 &  0.055766 \\
15 &  (3, elcajon, 15) &     (2, donovan) &  1.012843 &  1.253126 &   0.063455 &  0.055766 \\
29 &  (3, elcajon, 18) &     (1, shafter) &  1.067763 &  1.316716 &   0.066897 &  0.058408 \\
30 &  (3, elcajon, 18) &     (2, donovan) &  1.067763 &  1.316716 &   0.066897 &  0.058408 \\
28 &  (3, elcajon, 20) &     (2, donovan) &  1.001709 &  1.213527 &   0.062847 &  0.053727 \\
\bottomrule
\end{tabular}

\caption{Level 1 train results for Subu}
\end{table}
\begin{table}[H]
\centering
\scriptsize
\begin{tabular}{llrrrr}
\toprule
{} &             Model &    NO2 MAE &     O3 MAE &  NO2 CvMAE &  O3 CvMAE \\
\midrule
0  &  (3, donovan, 11) &   5.285262 &   9.102821 &   0.983236 &  0.240546 \\
1  &  (3, donovan, 12) &   4.855495 &   7.769970 &   0.903285 &  0.205325 \\
2  &  (1, shafter, 15) &   8.732711 &  16.912898 &   0.662886 &  0.557779 \\
3  &  (2, donovan, 20) &   9.083938 &  13.480725 &   0.569503 &  0.598717 \\
4  &  (1, elcajon, 13) &   6.490584 &   8.779205 &   0.551238 &  0.295096 \\
5  &  (1, shafter, 18) &  10.137307 &  30.810658 &   0.831440 &  1.066714 \\
6  &  (2, elcajon, 19) &   2.826961 &   6.512799 &   0.670877 &  0.171718 \\
7  &  (3, elcajon, 15) &   5.717879 &  10.318987 &   0.613544 &  0.283973 \\
8  &  (2, donovan, 15) &  11.550193 &  13.683201 &   1.138293 &  0.510384 \\
9  &  (1, elcajon, 11) &   6.017369 &   8.985467 &   0.505963 &  0.300680 \\
10 &  (1, donovan, 19) &   4.454034 &  13.303465 &   0.395614 &  0.347789 \\
11 &  (2, donovan, 18) &  11.489295 &  14.276486 &   1.181659 &  0.528417 \\
12 &  (1, donovan, 21) &   4.434544 &  11.105627 &   0.393883 &  0.290331 \\
13 &  (2, elcajon, 21) &   2.634243 &   5.934458 &   0.625142 &  0.156469 \\
14 &  (3, elcajon, 20) &   6.200108 &   8.415546 &   0.595342 &  0.197600 \\
15 &  (3, elcajon, 18) &   6.346586 &  11.165504 &   0.704758 &  0.304157 \\
16 &  (3, donovan, 13) &   6.243181 &  10.143651 &   1.161441 &  0.268050 \\
17 &  (1, elcajon, 12) &   6.140668 &   9.099901 &   0.516326 &  0.304516 \\
\bottomrule
\end{tabular}

\caption{Level 1 test results for Subu}
\end{table}

\begin{table}[H]
\centering
\scriptsize
\begin{tabular}{llrrrr}
\toprule
{} &                               Model &   NO2 MAE &    O3 MAE &  NO2 CvMAE &  O3 CvMAE \\
\midrule
0  &  (19, \{(3, shafter), (2, elcajon)\}) &  1.151536 &  1.419818 &   0.088122 &  0.052819 \\
1  &  (13, \{(2, shafter), (3, donovan)\}) &  1.370916 &  1.596229 &   0.110297 &  0.052533 \\
2  &  (12, \{(3, donovan), (1, elcajon)\}) &  1.374757 &  1.729670 &   0.128082 &  0.055116 \\
3  &  (19, \{(1, donovan), (3, shafter)\}) &  1.208029 &  1.606529 &   0.100613 &  0.063356 \\
4  &  (18, \{(3, elcajon), (2, donovan)\}) &  1.275981 &  1.494079 &   0.088022 &  0.053545 \\
5  &  (21, \{(3, shafter), (2, elcajon)\}) &  1.102141 &  1.430002 &   0.084342 &  0.053198 \\
6  &  (11, \{(2, shafter), (1, elcajon)\}) &  0.801864 &  1.291776 &   0.075264 &  0.038271 \\
7  &  (11, \{(2, shafter), (3, donovan)\}) &  1.291893 &  1.529065 &   0.103402 &  0.050457 \\
8  &  (12, \{(2, shafter), (3, donovan)\}) &  1.330020 &  1.651226 &   0.106763 &  0.054405 \\
9  &  (15, \{(3, elcajon), (2, donovan)\}) &  1.243857 &  1.477873 &   0.085535 &  0.052867 \\
10 &  (21, \{(1, donovan), (2, elcajon)\}) &  0.857213 &  1.410411 &   0.098307 &  0.036701 \\
11 &  (15, \{(2, donovan), (1, shafter)\}) &  1.145883 &  1.549991 &   0.130242 &  0.040771 \\
12 &  (19, \{(1, donovan), (2, elcajon)\}) &  0.874947 &  1.443156 &   0.100341 &  0.037553 \\
13 &  (15, \{(3, elcajon), (1, shafter)\}) &  0.982328 &  1.329509 &   0.071859 &  0.053218 \\
14 &  (11, \{(3, donovan), (1, elcajon)\}) &  1.374018 &  1.695738 &   0.127891 &  0.054094 \\
15 &  (12, \{(2, shafter), (1, elcajon)\}) &  0.827297 &  1.312652 &   0.077697 &  0.038959 \\
16 &  (13, \{(3, donovan), (1, elcajon)\}) &  1.337952 &  1.681822 &   0.126940 &  0.053464 \\
17 &  (18, \{(2, donovan), (1, shafter)\}) &  1.433064 &  1.757262 &   0.150534 &  0.043476 \\
18 &  (21, \{(1, donovan), (3, shafter)\}) &  1.118296 &  1.558826 &   0.093140 &  0.061475 \\
19 &  (18, \{(3, elcajon), (1, shafter)\}) &  1.106928 &  1.503584 &   0.073492 &  0.064203 \\
20 &  (13, \{(2, shafter), (1, elcajon)\}) &  0.823050 &  1.313720 &   0.077468 &  0.038788 \\
\bottomrule
\end{tabular}

\caption{Level 2 train results for NN[4]}
\end{table}
\begin{table}[H]
\centering
\scriptsize
\begin{tabular}{lrrrrr}
\toprule
{} &  Model &   NO2 MAE &     O3 MAE &  NO2 CvMAE &  O3 CvMAE \\
\midrule
0 &   18.0 &  6.597644 &   9.674718 &   0.633514 &  0.227166 \\
1 &   15.0 &  5.944096 &  12.223563 &   0.717022 &  0.371152 \\
2 &   15.0 &  7.954184 &  14.602328 &   0.500794 &  0.647935 \\
3 &   15.0 &  5.750527 &   9.380494 &   0.541098 &  0.218423 \\
4 &   18.0 &  5.987559 &  20.079483 &   0.375380 &  0.891772 \\
5 &   18.0 &  6.596008 &  13.337646 &   0.829702 &  0.396037 \\
\bottomrule
\end{tabular}

\caption{Level 2 test results for Subu}
\end{table}

\begin{table}[H]
\centering
\scriptsize
\begin{tabular}{lrrrrr}
\toprule
{} &  Model &   NO2 MAE &    O3 MAE &  NO2 CvMAE &  O3 CvMAE \\
\midrule
0 &   20.0 &  0.708947 &  0.853971 &   0.049104 &  0.030566 \\
1 &   21.0 &  0.484349 &  0.806330 &   0.056149 &  0.021130 \\
2 &   18.0 &  0.841794 &  1.032434 &   0.060138 &  0.036348 \\
3 &   19.0 &  0.500315 &  0.844828 &   0.058000 &  0.022139 \\
4 &   15.0 &  0.863321 &  1.109042 &   0.065332 &  0.038220 \\
5 &   17.0 &  0.392986 &  0.644915 &   0.034908 &  0.016909 \\
6 &   11.0 &  0.718655 &  0.859304 &   0.067949 &  0.027305 \\
7 &   12.0 &  0.733039 &  0.923635 &   0.069160 &  0.029355 \\
8 &   13.0 &  0.740070 &  0.926179 &   0.071228 &  0.029414 \\
\bottomrule
\end{tabular}

\caption{Level 3 train results for Subu}
\end{table}
\begin{table}[H]
\centering
\scriptsize
\begin{tabular}{lrrrrr}
\toprule
{} &  Model &   NO2 MAE &    O3 MAE &  NO2 CvMAE &  O3 CvMAE \\
\midrule
0 &   20.0 &  0.705669 &  0.833132 &   0.048590 &  0.029815 \\
1 &   21.0 &  0.499209 &  0.824901 &   0.057369 &  0.021700 \\
2 &   18.0 &  0.804261 &  0.979356 &   0.057286 &  0.034899 \\
3 &   19.0 &  0.510522 &  0.838838 &   0.058669 &  0.022067 \\
4 &   15.0 &  0.851035 &  1.147736 &   0.064501 &  0.039649 \\
5 &   17.0 &  0.816378 &  1.320163 &   0.072582 &  0.034112 \\
6 &   11.0 &  0.721102 &  0.866295 &   0.066733 &  0.027684 \\
7 &   12.0 &  0.742562 &  0.913975 &   0.069302 &  0.029184 \\
8 &   13.0 &  0.759674 &  0.929994 &   0.072043 &  0.029657 \\
\bottomrule
\end{tabular}

\caption{Level 3 test results for Subu}
\end{table}

\end{document}
